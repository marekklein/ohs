\nadpisleft{Lectio 6}

Exacuérunt tamquam gládium linguas suas. Non dicant Judæi: Non occídimus Christum. Etenim proptérea eum dedérunt júdici Pilato, ut quasi ipsi a morte ejus videréntur immunes. Nam cum dixísset eis Pilátus: Vos eum occídite: respondérunt, Nobis non licet occídere quemquam. Iniquitátem facínoris sui in judicem hóminem refúndere volébant: sed numquid Deum judicem fallebant? Quod fecit Pilatus, in eo ipso quod fecit, aliquantum particeps fuit: sed in comparatióne illórum multo ipse innocentior. Institit enim quantum pótuit, ut illum ex eórum mánibus liberaret: nam proptérea flagellátum produxit ad eos. Non persequéndo Dóminum flagellávit, sed eórum furori satisfácere volens: ut vel sic jam mitéscerent, et desínerent velle occídere, cum flagellátum víderent. Fecit et hoc. At ubi perseveravérunt, nostis illum lavisse manus, et dixisse, quod ipse non fecísset, mundum se esse a morte illíus. Fecit tamen. Sed si reus, quia fecit vel invítus: illi innocéntes, qui coëgérunt ut fáceret? Nullo modo. Sed ille dixit in eum senténtiam, et jussit eum crucifígi, et quasi ipse occídit: et vos, o Judæi, occídistis. Unde occídistis? Gladio linguæ: acuístis enim linguas vestras. Et quando percússistis, nisi quando clamastis: Crucifíge, crucifíge?
\par