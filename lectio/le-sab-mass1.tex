\nadpisleft{Lectio 1 (Gen 1:1--2:2)}

Léctio libri Génesis

In princípio creávit Deus cælum et terram. Terra autem erat inánis et vácua, et ténebræ super fáciem abýssi, et spíritus Dei ferebátur super aquas. 

Dixítque Deus: «Fiat lux». Et facta est lux.
Et vidit Deus lucem quod esset bona et divísit Deus lucem ac ténebras. 
Appellavítque Deus lucem Diem et ténebras Noctem.
Factúmque est véspere et mane, dies unus.

Dixit quoque Deus: «Fiat firmaméntum in médio aquárum et dívidat aquas ab aquis». Et fecit Deus firmaméntum divisítque aquas, quæ erant sub firmaménto, ab his, quæ erant super firmaméntum. Et factum est ita.
Vocavítque Deus firmaméntum Cælum. Et factum est véspere et mane, dies secúndus.

Dixit vero Deus: «Congregéntur aquæ, quæ sub cælo sunt, in locum unum, et appáreat árida». Factúmque est ita. 
Et vocávit Deus áridam Terram congregationésque aquárum appellávit Mária. Et vidit Deus quod esset bonum.

Et ait Deus: «Gérminet terra herbam viréntem et herbam faciéntem semen et lignum pomíferum fáciens fructum iuxta genus suum, cuius semen in semetípso sit super terram». Et factum est ita.
Et prótulit terra herbam viréntem et herbam afferéntem semen iuxta genus suum lignúmque fáciens fructum, qui habet in semetípso seméntem secúndum spéciem suam. Et vidit Deus quod esset bonum. Et factum est véspere et mane, dies tértius.

Dixit autem Deus: «Fiant luminária in firmaménto cæli, ut dívidant diem ac noctem et sint in signa et témpora et dies et annos, ut lúceant in firmaménto cæli et illúminent terram». Et factum est ita. 
Fecítque Deus duo magna luminária: lumináre maius, ut præésset diéi, et lumináre minus, ut præésset nocti, et stellas.
Et pósuit eas Deus in firmaménto cæli, ut lucérent super terram et præéssent diéi ac nocti et divíderent lucem ac ténebras. Et vidit Deus quod esset bonum.
Et factum est véspere et mane, dies quartus.

Dixit étiam Deus: «Púllulent aquæ réptile ánimæ vivéntis, et volátile volet super terram sub firmaménto cæli». Creavítque Deus cete grándia et omnem ánimam vivéntem atque motábilem, quam púllulant aquæ secúndum spécies suas, et omne volátile secúndum genus suum. Et vidit Deus quod esset bonum; benedixítque eis Deus dicens: «Créscite et multiplicámini et repléte aquas maris, avésque multiplicéntur super terram». Et factum est véspere et mane, dies quintus.

Dixit quoque Deus: «Prodúcat terra ánimam vivéntem in génere suo, iuménta et reptília et béstias terræ secúndum spécies suas».
Factúmque est ita. Et fecit Deus béstias terræ iuxta spécies suas et iuménta secúndum spécies suas et omne réptile terræ in génere suo. Et vidit Deus quod esset bonum.

Et ait Deus: «Faciámus hóminem ad imáginem et similitúdinem nostram; et præsint píscibus maris et volatílibus cæli et béstiis univers\'{\ae}que terræ omníque réptili, quod movétur in terra».
Et creávit Deus hóminem ad imáginem suam; ad imáginem Dei creávit illum; másculum et féminam creávit eos. 

Benedixítque illis Deus et ait illis Deus: «Créscite et multiplicámini et repléte terram et subícite eam et dominámini píscibus maris et volatílibus cæli et univérsis animántibus, quæ movéntur super terram».
Dixítque Deus: «Ecce dedi vobis omnem herbam afferéntem semen super terram et univérsa ligna, quæ habent in semetípsis fructum ligni portántem seméntem, ut sint vobis in escam et cunctis animántibus terræ omníque vólucri cæli et univérsis, quæ movéntur in terra et in quibus est ánima vivens, omnem herbam viréntem ad vescéndum». Et factum est ita.
Vidítque Deus cuncta, quæ fecit, et ecce erant valde bona. Et factum est véspere et mane, dies sextus.

Igitur perfécti sunt cæli et terra et omnis exércitus eórum. 
Complevítque Deus die séptimo opus suum, quod fécerat, et requiévit die séptimo
ab univérso ópere, quod patrárat.

\textbf{\Vbar.} Verbum Dómini.
\textbf{\Rbar.} Deo grátias.
\par