\nadpisleft{Lectio 5 (Is 55:1-11)}

Léctio libri Isaíæ prophétæ

Hæc dicit Dóminus: 

«Omnes sitiéntes, veníte ad aquas; et, qui non habétis argéntum, properáte, émite et comédite, veníte, émite absque argénto et absque ulla commutatióne vinum et lac.
Quare appénditis argéntum non in pánibus et labórem vestrum non in saturitáte? Audíte, audiéntes me, et comédite bonum, ut delectétur in crassitúdine ánima vestra.

Inclináte aurem vestram et veníte ad me; audíte, ut vivat ánima vestra, et fériam vobíscum pactum sempitérnum, misericórdias David fidéles.
Ecce testem pópulis dedi eum, ducem ac præceptórem géntibus.
Ecce gentem, quam nesciébas, vocábis, et gentes, quæ te non cognovérunt, ad te current, propter Dóminum Deum tuum et Sanctum Israel, quia glorificávit te.

Qu\'{\ae}rite Dóminum, dum inveníri potest; invocáte eum, dum prope est.
Derelínquat ímpius viam suam, et vir iníquus cogitatiónes suas; et revertátur ad Dóminum, et miserébitur eius, et ad Deum nostrum, quóniam multus est ad ignoscéndum.
Non enim cogitatiónes meæ cogitatiónes vestræ, neque viæ vestræ viæ meæ, dicit Dóminus.
Quia sicut exaltántur cæli a terra, sic exaltátæ sunt viæ meæ a viis vestris et cogitatiónes meæ a cogitatiónibus vestris.

Et quómodo descéndit imber et nix de cælo et illuc ultra non revértitur, sed inébriat terram et infúndit eam et germináre eam facit et dat semen serénti et panem comedénti, sic erit verbum meum, quod egrediétur de ore meo: non revertétur ad me vácuum, sed fáciet, quæcúmque vólui, et prosperábitur in his, ad quæ misi illud.

\textbf{\Vbar.} Verbum Dómini.
\textbf{\Rbar.} Deo grátias.
\par