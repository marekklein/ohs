\nadpisleft{Evangelium (Io. 13, 1-15)}

\textbf{\Vbar.} Dóminus vobíscum.
\textbf{\Rbar.} Et cum spíritu tuo.

\textbf{\Vbar.} Lectio sancti Evangélii secúndum Joánnem.
\textbf{\Rbar.} Glória tibi, Dómine.

Ante diem festum Paschæ, sciens Iesus quia venit eius hora, ut tránseat ex hoc mundo ad Patrem, cum dilexísset suos, qui erant in mundo, in finem diléxit eos.

Et in cena, cum Diábolus iam misísset in corde, ut tráderet eum Iudas Simónis Iscariótis, sciens quia ómnia dedit ei Pater in manus et quia a Deo exívit et ad Deum vadit, surgit a cena et ponit vestiménta sua et, cum accepísset línteum, præcínxit se. Deínde mittit aquam in pelvem et cœpit laváre pedes discipulórum et extérgere línteo, quo erat præcínctus. 

Venit ergo ad Simónem Petrum. Dicit ei: «Dómine, tu mihi lavas pedes?».
Respóndit Iesus et dixit ei: «Quod ego fácio, tu nescis modo, scies autem póstea». 
Dicit ei Petrus: «Non lavábis mihi pedes in ætérnum!». 
Respóndit Iesus ei: «Si non lávero te, non habes partem mecum». 
Dicit ei Simon Petrus: «Dómine, non tantum pedes meos, sed et manus et caput!».
Dicit ei Iesus: «Qui lotus est, non índiget nisi ut pedes lavet, sed est mundus totus; et vos mundi estis, sed non omnes».
Sciébat enim quisnam esset, qui tráderet eum; proptérea dixit:
«Non estis mundi omnes». 

Postquam ergo lavit pedes eórum et accépit vestiménta sua, cum recubuísset íterum, dixit eis: «Scitis quid fécerim vobis? 
Vos vocátis me: “Magíster”, et: “Dómine”, et bene dícitis; sum étenim.
Si ergo ego lavi vestros pedes, Dóminus et Magíster, et vos debétis alter altérius laváre pedes.
Exémplum enim dedi vobis, ut, quemádmodum ego feci vobis, et vos faciátis».

\textbf{\Vbar.} Verbum Dómini.
\textbf{\Rbar.} Deo grátias.
\par