\nadpisleft{Lectio 1 (Io 18:1--19:42)}

Passio Dómini nostri Iesu Christi secúndum Ioánnem

In illo témpore : Egréssus Iesus cum discípulis suis trans torréntem Cedron, ubi erat hortus, in quem introívit ipse et discípuli eíus. Sciébat autem et Iúdas, qui tradébat eum, locum, quia frequénter Iésus convénerat illuc cum discípulis suis. Iúdas ergo, cum accepísset cohórtem et a pontifícibus et pharisǽis minístros, venit illuc cum lantérnis et fácibus et armis. Iésus ítaque sciens ómnia, quae ventúra erant super eum, procéssit et dicit eis: " Quem quǽritis? ". Respondérunt ei: " Iésum Nazarénum ". Dicit eis: " Ego sum! ". Stabat autem et Iúdas, qui tradébat eum, cum ipsis. Ut ergo dixit eis: " Ego sum! ", abiérunt retrórsum et cecidérunt in terram. Íterum ergo eos interrogávit: " Quem quǽritis? ". Illi autem dixérunt: " Iésum Nazarénum ". Respóndit Iésus: " Dixi vobis: Ego sum! Si ergo me quǽritis, sínite hos abíre ", ut implerétur sermo, quem dixit: " Quos dedísti mihi, non pérdidi ex ipsis quemquam ". Simon ergo Petrus, habens gládium, edúxit eum et percússit pontíficis servum et abscídit eíus aurículam dextram. Erat autem nomen servo Malchus. Dixit ergo Iésus Petro: " Mitte gládium in vagínam; cálicem, quem dedit mihi Pater, non bibam illum? ". Cohors ergo et tribúnus et minístri Iudaeórum comprehendérunt Iésum et ligavérunt eum et adduxérunt ad Annam primum; erat enim socer Cáiphae, qui erat póntifex anni illíus. Erat autem Cáiphas, qui consílium déderat Iudǽis: " Éxpedit unum hóminem mori pro pópulo ". Sequebátur autem Iésum Simon Petrus et álius discípulus. Discípulus autem ille erat notus pontífici et introívit cum Iésu in átrium pontíficis; Petrus autem stabat ad óstium foris. Exívit ergo discípulus álius, qui erat notus pontífici, et dixit ostiáriae et introdúxit Petrum. Dicit ergo Petro ancílla ostiária: " Numquid et tu ex discípulis es hóminis istíus? ". Dicit ille: " Non sum! ". Stabant autem servi et minístri, qui prunas fécerant, quia frigus erat, et calefaciébant se; erat autem cum eis et Petrus stans et calefáciens se. Póntifex ergo interrogávit Iésum de discípulis suis et de doctrína eíus. Respóndit ei Iésus: " Ego palam locútus sum mundo; ego semper dócui in synagóga et in templo, quo omnes Iudǽi convéniunt, et in occúlto locútus sum nihil.  Quid me intérrogas? Intérroga eos, qui audiérunt quid locútus sum ipsis; ecce hi sciunt, quae díxerim ego ".  Haec autem cum dixísset, unus assístens ministrórum dedit álapam Iésu dicens: " Sic respóndes pontífici? ". Respóndit ei Iésus: " Si male locútus sum, testimónium pérhibe de malo; si autem bene, quid me caedis? ".  Misit ergo eum Annas ligátum ad Cáipham pontíficem. Erat autem Simon Petrus stans et calefáciens se. Dixérunt ergo ei: " Numquid et tu ex discípulis eíus es? ". Negávit ille et dixit: " Non sum! ". Dicit unus ex servis pontíficis, cognátus eíus, cuíus abscídit Petrus aurículam: " Nonne ego te vidi in horto cum illo? ".  Íterum ergo negávit Petrus; et statim gallus cantávit. Addúcunt ergo Iésum a Cáipha in praetórium. Erat autem mane. Et ipsi non introiérunt in praetórium, ut non contaminaréntur, sed manducárent Pascha. Exívit ergo Pilátus ad eos foras et dicit: " Quam accusatiónem affértis advérsus hóminem hunc? ". Respondérunt et dixérunt ei: " Si non esset hic malefáctor, non tibi tradidissémus eum ".  Dixit ergo eis Pilátus: " Accípite eum vos et secúndum legem vestram iudicáte eum! ". Dixérunt ei Iudǽi: " Nobis non licet interfícere quemquam ", ut sermo Iésu implerétur, quem dixit, signíficans qua esset morte moritúrus. Introívit ergo íterum in praetórium Pilátus et vocávit Iésum et dixit ei: " Tu es rex Iudaeórum? ". Respóndit Iésus: " A temetípso tu hoc dicis, an álii tibi dixérunt de me? ". Respóndit Pilátus: " Numquid ego Iudǽus sum? Gens tua et pontífices tradidérunt te mihi; quid fecísti? ". Respóndit Iésus: " Regnum meum non est de mundo hoc; si ex hoc mundo esset regnum meum, minístri mei decertárent, ut non tráderer Iudǽis; nunc autem meum regnum non est hinc ". Dixit ítaque ei Pilátus: " Ergo rex es tu? ". Respóndit Iésus: " Tu dicis quia rex sum. Ego in hoc natus sum et ad hoc veni in mundum, ut testimónium perhíbeam veritáti; omnis, qui est ex veritáte, audit meam vocem ". Dicit ei Pilátus: " Quid est véritas? ". Et cum hoc dixísset, íterum exívit ad Iudǽos et dicit eis: " Ego nullam invénio in eo causam. Est autem consuetúdo vobis, ut unum dimíttam vobis in Pascha; vultis ergo dimíttam vobis regem Iudaeórum? ". Clamavérunt ergo rursum dicéntes: " Non hunc sed Barábbam! ". Erat autem Barábbas latro. Tunc ergo apprehéndit Pilátus Iésum et flagellávit. Et mílites, plecténtes corónam de spinis, imposuérunt cápiti eíus et veste purpúrea circumdedérunt eum; et veniébant ad eum et dicébant: " Ave, rex Iudaeórum! ", et dabant ei álapas. Et éxiit íterum Pilátus foras et dicit eis: " Ecce addúco vobis eum foras, ut cognoscátis quia in eo invénio causam nullam ". Éxiit ergo Iésus foras, portans spíneam corónam et purpúreum vestiméntum. Et dicit eis: " Ecce homo! ". Cum ergo vidíssent eum pontífices et minístri, clamavérunt dicéntes: " Crucifíge, crucifíge! ". Dicit eis Pilátus: " Accípite eum vos et crucifígite; ego enim non invénio in eo causam ". Respondérunt ei Iudǽi: " Nos legem habémus, et secúndum legem debet mori, quia Fílium Dei se fecit ". Cum ergo audísset Pilátus hunc sermónem, magis tímuit et ingréssus est praetórium íterum et dicit ad Iésum: " Unde es tu? ". Iésus autem respónsum non dedit ei. Dicit ergo ei Pilátus: " Mihi non lóqueris? Nescis quia potestátem hábeo dimíttere te et potestátem hábeo crucifígere te? ". Respóndit Iésus: " Non habéres potestátem advérsum me ullam, nisi tibi esset datum désuper; proptérea, qui trádidit me tibi, maíus peccátum habet ". Exínde quaerébat Pilátus dimíttere eum; Iudǽi autem clamábant dicéntes: " Si hunc dimíttis, non es amícus Cǽsaris! Omnis, qui se regem facit, contradícit Cǽsari ". Pilátus ergo, cum audísset hos sermónes, addúxit foras Iésum et sedit pro tribunáli in locum, qui dícitur Lithóstrotos, Hebráice autem Gábbatha. Erat autem Parascéve Paschae, hora erat quasi sexta. Et dicit Iudǽis: " Ecce rex vester! ". Clamavérunt ergo illi: " Tolle, tolle, crucifíge eum! ". Dicit eis Pilátus: " Regem vestrum crucifígam? ". Respondérunt pontífices: " Non habémus regem, nisi Cǽsarem ". Tunc ergo trádidit eis illum, ut crucifigerétur. Suscepérunt ergo Iésum. Et baíulans sibi crucem exívit in eum, qui dícitur Calváriae locum, quod Hebráice dícitur Gólgotha, ubi eum crucifixérunt et cum eo álios duos hinc et hinc, médium autem Iésum. Scripsit autem et títulum Pilátus et pósuit super crucem; erat autem scriptum: " Iésus Nazarénus Rex Iudaeórum ". Hunc ergo títulum multi legérunt Iudaeórum, quia prope civitátem erat locus, ubi crucifíxus est Iésus; et erat scriptum Hebráice, Latíne, Graece. Dicébant ergo Piláto pontífices Iudaeórum: " Noli scríbere: Rex Iudaeórum, sed: Ipse dixit: "Rex sum Iudaeórum" ". Respóndit Pilátus: " Quod scripsi, scripsi! ". Mílites ergo cum crucifixíssent Iésum, accepérunt vestiménta eíus et fecérunt quáttuor partes, unicuíque míliti partem, et túnicam. Erat autem túnica inconsútilis, désuper contéxta per totum. Dixérunt ergo ad ínvicem: " Non scindámus eam, sed sortiámur de illa,cuíus sit ", ut Scriptúra impleátur dicens: " Partíti sunt vestiménta mea sibi et in vestem meam misérunt sortem ". Et mílites quidem haec fecérunt.  Stabant autem iúxta crucem Iésu mater eíus et soror matris eíus, María Cléopae, et María Magdaléne. Cum vidísset ergo Iésus matrem et discípulum stantem, quem diligébat, dicit matri: " Múlier, ecce fílius tuus ". Deínde dicit discípulo: " Ecce mater tua ". Et ex illa hora accépit eam discípulus in sua. Post hoc sciens Iésus quia iam ómnia consummáta sunt, ut consummarétur Scriptúra, dicit: " Sítio ". Vas pósitum erat acéto plenum; spóngiam ergo plenam acéto hyssópo circumponéntes, obtulérunt ori eíus. Cum ergo accepísset acétum, Iésus dixit: " Consummátum est! ". Et inclináto cápite trádidit spíritum.

\begin{footnotesize}\textit{\color{red} Hic genuflectitur, et pausatur aliquantulum.}\end{footnotesize}

Ici on génuflecte, et on fait une petite pause.
Iudǽi ergo, quóniam Parascéve erat, ut non remanérent in cruce córpora sábbato, erat enim magnus dies illíus sábbati, rogavérunt Pilátum, ut frangeréntur eórum crura, et tolleréntur. Venérunt ergo mílites et primi quidem fregérunt crura et altérius, qui crucifíxus est cum eo; ad Iésum autem cum veníssent, ut vidérunt eum iam mórtuum, non fregérunt eíus crura, sed unus mílitum láncea latus eíus apéruit, et contínuo exívit sanguis et aqua. Et qui vidit, testimónium perhíbuit, et verum est eíus testimónium, et ille scit quia vera dicit, ut et vos credátis. Facta sunt enim haec, ut Scriptúra impleátur: " Os non comminuétur eíus ", et íterum ália Scriptúra dicit: " Vidébunt in quem transfixérunt ". Post hæc autem rogávit Pilátum Ióseph ab Arimathǽa, qui erat discípulus Iésu, occúltus autem propter metum Iudaeórum, ut tólleret corpus Iésu; et permísit Pilátus. Venit ergo et tulit corpus eíus. Venit autem et Nicodémus, qui vénerat ad eum nocte primum, ferens mixtúram myrrhae et aloes quasi libras centum. Accepérunt ergo corpus Iésu et ligavérunt illud línteis cum aromátibus, sicut mos Iudǽis est sepelíre. Erat autem in loco, ubi crucifíxus est, hortus, et in horto monuméntum novum, in quo nondum quisquam pósitus erat.
\par
