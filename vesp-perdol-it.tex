\ohschapter{Ad Vesperas}
%\vspace{-4em}
%\nadpisd{cum psalmi iuxta vulgata nova}
\vspace{1em}

\redvbar Bože, príď mi na pomoc.

\redrbar Pane, ponáhľaj sa mi pomáhať. Sláva Otcu i Synu i Duchu Svätému. Ako bolo na počiatku, tak nech je i teraz i vždycky, i na veky vekov. Amen. Aleluja.
%\ohsformatverses
%\gresetinitiallines{1}%
%\grechangedim{annotationraise}{-2 mm}{scalable}
%\greannotation{\color{red} \Vbar}
%\gregorioscore{./other/deus-in-adjutorium-alleluia}
%%%%%%%%%%%%%%%%%%%%%%%%%%%%%%%%%%%%%%%%%%%%%%%%%%%%
\vspace{2em}

\nadpisb{Hymnus}
\hfill\begin{minipage}{\dimexpr\textwidth-2.4cm}
Ó, Mária bolestivá, naša ochrana,\newline 
slovenský náš národ volá: Pros za nás Boha.\newline 
[:Ty si Mať dobrotivá, Patrónka ľútostivá,\newline 
oroduj vždy za náš národ u svojho Syna.:]

\vspace{1em}

Matka drahá, spomni na nás tu pod Tatrami,\newline 
verní sme my svätej Cirkvi, prebývaj s nami!\newline 
[:Ty si Mať dobrotivá, Patrónka ľútostivá,\newline 
oroduj vždy za náš národ u svojho Syna.:]
\vspace{1em}

Po horách i po dolinách venčíme tvoj chrám,\newline 
pútne miesta volajú ti: Matka, národ chráň!\newline
[:Ty si Mať dobrotivá, Patrónka ľútostivá,\newline 
oroduj vždy za náš národ u svojho Syna.:]
%\xdef\tpd{\the\prevdepth}
\end{minipage}


%%%%%%%%%%%%%%%%%%%%%%%%%%%%%%%%%%%%%%%%%%%%%%%%%%%%
%\vspace{2em}
\pagebreak
\nadpisb{Psalmódia}
\ohsformatantiphona
\ohsannotation{Ant. 1}{II d}
\gregorioscore{./ant/ant-a_planta_pedis}
\ohstranslate{./ant-a_planta_pedis}
\small{\ohspsalmus{Žalm 122}{./psalmivn/ps122sk}{./psalmivn/ps122sk.tex}}
\vskip 1mm%
{\color{red}
\centerline{\rule{0.6\linewidth}{1.4pt}}
}
\begin{flushleft}
\begin{enumerate}[leftmargin=*]
%\setcounter{enumi}{1}
\item Quale gioia, quando mi dissero: * \mbox{«Andremo alla casa del Signore».}
\item E ora i nostri piedi si fermano * \mbox{alle tue porte, Gerusalemme!}
\item Gerusalemme è costruita * \mbox{come città salda e compatta.}
\item Là salgono insieme le tribù, le tribù del Signore, † secondo la legge di Israele, * \mbox{per lodare il nome del Signore.}
\item Là sono posti i seggi del giudizio, * \mbox{i seggi della casa di Davide.}
\item Domandate pace per Gerusalemme: * \mbox{sia pace a coloro che ti amano,}
\item sia pace sulle tue mura, * \mbox{sicurezza nei tuoi baluardi.}
\item Per i miei fratelli e i miei amici * \mbox{io dirò: «Su di te sia pace!».}
\item Per la casa del Signore nostro Dio, * \mbox{chiederò per te il bene.}
\item Gloria al Padre e al Figlio * \mbox{e allo Spirito Santo.}
\item Come era nel principio, e ora e sempre * \mbox{nei secoli dei secoli. Amen.}
\end{enumerate}
\end{flushleft}


\ohsscorenostar{./ant/ant-a_planta_pedis}
\separator{3}

%%%%%%%%%%%%%%%%%%%%%%%%%%%%%%%%%%%%%%%%%%%%%%%%%%%%
\ohsformatantiphona
\ohsannotation{Ant. 2}{II d}
\gregorioscore{./ant/ant-oppresssit_me_dolor}
\ohstranslate{./ant-oppresssit_me_dolor}
\small{\ohspsalmus{Žalm 127}{./psalmivn/ps127sk}{./psalmivn/ps127sk.tex}}
\input{./psalmivn/ps127it.tex}
\ohsscorenostar{./ant/ant-oppresssit_me_dolor}
\separator{3}

%%%%%%%%%%%%%%%%%%%%%%%%%%%%%%%%%%%%%%%%%%%%%%%%%%%%
\ohsformatantiphona
\ohsannotation{Ant. 3}{I a}
\gregorioscore{./ant/ant-dominus_dedit}
\ohstranslate{./ant-dominus_dedit}
\small{\ohspsalmus{Chválospev Ef1, 3-10}{./psalmivn/ef1sk}{./psalmivn/ef1sk.tex}}
\vskip 1mm%
{\color{red}
\centerline{\rule{0.6\linewidth}{1.4pt}}
}
\begin{flushleft}
\begin{enumerate}[leftmargin=*]
%\setcounter{enumi}{1}
\item Benedetto sia Dio, Padre del Signore nostro Gesù Cristo, * che ci ha benedetti con ogni benedizione spirituale nei cieli, in Cristo.
\item In lui ci ha scelti * \mbox{prima della creazione del mondo,}
\item per trovarci, al suo cospetto, * \mbox{santi e immacolati nell’amore.}
\item Ci ha predestinati * \mbox{a essere suoi figli adottivi}
\item per opera di Gesù Cristo, * \mbox{secondo il beneplacito del suo volere,}
\item a lode e gloria della sua grazia, * \mbox{che ci ha dato nel suo Figlio diletto.}
\item In lui abbiamo la redenzione mediante il suo sangue, * \mbox{la remissione dei peccati secondo la ricchezza della sua grazia.}
\item Dio l’ha abbondantemente riversata su di noi con ogni  sapienza e intelligenza, * \mbox{poiché egli ci ha fatto conoscere il mistero del suo volere,}
\item il disegno di ricapitolare in Cristo tutte le cose, * \mbox{quelle del cielo come quelle della terra.}
\item Nella sua benevolenza lo aveva in lui prestabilito * \mbox{per realizzarlo nella pienezza dei tempi.}
\item Gloria al Padre e al Figlio * \mbox{e allo Spirito Santo.}
\item Come era nel principio, e ora e sempre * \mbox{nei secoli dei secoli. Amen.}
\end{enumerate}
\end{flushleft}


\ohsscorenostar{./ant/ant-dominus_dedit}
\separator{3}

%%%%%%%%%%%%%%%%%%%%%%%%%%%%%%%%%%%%%%%%%%%%%%%%%%%%
\vspace{1em}
\nadpisb{Krátke čítanie}
\nadpisleft{Gal 4, 4-7} 

Keď prišla plnosť času, Boh poslal svojho Syna, narodeného zo ženy, narodeného pod zákonom, 
aby vykúpil tých, čo boli pod zákonom, a aby sme dostali adoptívne synovstvo. 
Pretože ste synmi, poslal Boh do našich sŕdc Ducha svojho Syna a on volá: „Abba, Otče!“ 
A tak už nie si otrok, ale syn; a keď syn, tak skrze Boha aj dedič.


%%%%%%%%%%%%%%%%%%%%%%%%%%%%%%%%%%%%%%%%%%%%%%%%%%%%

%%%%%%%%%%%%%%%%%%%%%%%%%%%%%%%%%%%%%%%%%%%%%%%%%%%%
%\vspace{1em}
\pagebreak
\nadpisb{Krátke responzórium}
\redvbar Stála svätá Mária, Kráľovná neba a Pani sveta pri Pánovom kríži.

\redrbar Stála svätá Mária, Kráľovná neba a Pani sveta pri Pánovom kríži.

\redvbar Šťastná je, veď bez smrti získala palmu mučeníctva.

\redrbar Pri Pánovom kríži.

\redvbar Sláva Otcu i Synu i Duchu Svätému.

\redrbar Stála svätá Mária, Kráľovná neba a Pani sveta pri Pánovom kríži.

%%%%%%%%%%%%%%%%%%%%%%%%%%%%%%%%%%%%%%%%%%%%%%%%%%%%
%\vspace{1em}
\separator{3}
\nadpisb{Evanjeliový chválospev}

%%%%%%%%%%%%%%%%%%%%%%%%%%%%%%%%%%%%%%%%%%%%%%%%%%%%
%%%%%%%%%%%%%%%%%%%%%%%%%%%%%%%%%%%%%%%%%%%%%%%%%%%%
\ohsformatantiphona
\ohsannotation{Ant.}{VIII g}
\gregorioscore{./ant/ant-cum_vidisset}
\ohstranslate{./ant-cum_vidisset}

\small{\ohspsalmus{Lk 1, 46-55}{./psalmivn/magnificat-sk}{./psalmivn/magnificat-sk.tex}}
\vskip 1mm%
{\color{red}
\centerline{\rule{0.6\linewidth}{1.4pt}}
}
\begin{flushleft}
\begin{enumerate}[leftmargin=*]
%\setcounter{enumi}{1}
\item L’anima mia magnifica il Signore * \mbox{e il mio spirito esulta in Dio, mio salvatore,}
\item perché ha guardato l’umiltà della sua serva. * \mbox{D’ora in poi tutte le generazioni mi chiameranno beata.}
\item Grandi cose ha fatto in me l’Onnipotente * \mbox{e Santo è il suo nome:}
\item di generazione in generazione la sua misericordia * \mbox{si stende su quelli che lo temono.}
\item Ha spiegato la potenza del suo braccio, * \mbox{ha disperso i superbi nei pensieri del loro cuore;}
\item ha rovesciato i potenti dai troni, * \mbox{ha innalzato gli umili;}
\item ha ricolmato di beni gli affamati, * \mbox{ha rimandato i ricchi a mani vuote.}
\item Ha soccorso Israele, suo servo, * \mbox{ricordandosi della sua misericordia,}
\item come aveva promesso ai nostri padri, * \mbox{ad Abramo e alla sua discendenza, per sempre.}
\item Gloria al Padre e al Figlio * \mbox{e allo Spirito Santo.}
\item Come era nel principio, e ora e sempre * \mbox{nei secoli dei secoli. Amen.}
\end{enumerate}
\end{flushleft}


\ohsscorenostar{./ant/ant-cum_vidisset}
\separator{3}

%%%%%%%%%%%%%%%%%%%%%%%%%%%%%%%%%%%%%%
\nadpisb{Prosby}
\small{
Z celého srdca chváľme všemohúceho Boha Otca, lebo z jeho vôle oslavujú Pannu Máriu, matku jeho Syna, všetky pokolenia, a pokorne ho prosme:

\textit{Nech oroduje za nás Sedembolestná.}

Bože, ty konáš vo svete veľké veci: poníženú Pannu Máriu si ustanovil za matku Cirkvi; 
daj, nech ju blahoslavia všetky pokolenia.

\redrbar Nech oroduje za nás Sedembolestná.

Ty si nám dal Pannu Máriu za matku a patrónku;

na jej orodovanie udeľ nášmu národu pokoj a spásu.

\redrbar Nech oroduje za nás Sedembolestná.

Ty si posilňoval Pannu Máriu, keď stála pod krížom, a pri vzkriesení svojho syna si ju naplnil radosťou;

uľahči trpiacim a posilni ich nádej.

\redrbar Nech oroduje za nás Sedembolestná.

Ty si Pannu Máriu urobil pozornou voči tvojmu slovu a svojou vernou služobnicou;

na jej príhovor urob z našich biskupov a kňazov verných služobníkov tvojho Syna a správcov jeho tajomstiev.

\redrbar Nech oroduje za nás Sedembolestná.

Ty si dal Panne Márii plnosť milosti;

daruj nám vytrvalosť vo viere a láske.

\redrbar Nech oroduje za nás Sedembolestná.

Ty si bolestnú Pannu korunoval za Kráľovnú neba;

daj, nech sa naši zosnulí naveky radujú so všetkými svätými v tvojom kráľovstve.

\redrbar Nech oroduje za nás Sedembolestná.}

%\separator{4}
%\pagebreak
%%%%%%%%%%%%%%%%%%%%%%%%%%%%%%%%%%%%%%
\vspace{1em}

\redvbar Præcéptis salutáribus móniti et divína institutióne formáti audémus dícere:

\redrbar Pater noster, qui es in cælis: 
sanctificétur nomen tuum;
advéniat regnum tuum;
fiat volúntas tua, sicut in cælo et in terra.
Panem nostrum cotidiánum da nobis hódie;
et dimítte nobis débita nostra,
sicut et nos dimíttimus debitóribus nostris;
et ne nos indúcas in tentatiónem;
sed líbera nos a malo.

\redvbar Deus, qui Fílio tuo in cruce exaltáto compatiéntem Matrem astáre voluísti, da Ecclésiæ tuæ, ut, Christi passiónis cum ipsa consors effécta, eiúsdem resurrectiónis párticeps esse mereátur. qui tecum vivit et regnat in unitáte Spíritus Sancti, Deus, per ómnia saécula saeculórum.
\redrbar Amen.

%\vspace{2em}
\pagebreak
\redvbar Dóminus vobíscum.

\redrbar Et cum spíritu tuo.

\vspace{1em}
\redvbar Sit nomen Dómini benedíctum.

\redrbar Ex hoc nunc et usque in saéculum.

\redvbar Adiutórium nostrum in nómine Dómini.

\redrbar Qui fecit cælum et terram.

\redvbar Benedícat vos omnípotens Deus, Pater, et Fílius, et Spíritus Sanctus.

\redrbar Amen.

\vspace{1em}
\redvbar Ite in pace.

\redrbar Deo grátias.
\vspace{3em}
\begin{center}\greseparator{2}{20}\end{center}

%\clearpage
\vspace*{\stretch{4}}
\begin{center}
%\begin{minipage}{.6\textwidth}
\small{Antifóny zodpovedajú pokynom \textit{ORDO CANTUS OFFICII}
\linebreak(Typis Vaticanis 2015)

Slovenské texty sa zhodujú so schváleným vydaním - slovenským prekladom \textit{Liturgia hodín podľa Rímskeho obradu} 
\linebreak(Typis Polyglottis Vaticanis, 1986)

V spolupráci s ThDr. Jozefom Kmecom, PhD.
\linebreak zostavil v roku 2023 Marek Klein (www.gregoriana.sk)}

%\end{minipage}
\end{center}
%\vspace{\stretch{1}}



%%%%%%%%%%%%%%%%%%%%%%%%%%%%%%%%%%%%%%%%%%%%%%%%%%%%%%%%%%%%%%%%%%%%%%%%%%%%%%%%%%%%%%%%%%%%%%%%

