%\versseparator
Avant la fête de Pâque, Jésus sachant que son heure était venue de passer de ce monde à son Père, comme il avait aimé les siens qui étaient dans le monde, il les aima jusqu’à l’excès. Et après le souper, le démon ayant déjà mis dans le cœur de Judas Iscariote, fils de Simon, le dessein de le trahir, Jésus, qui savait que son Père lui avait donné tout pouvoir, qu’il était sorti de Dieu et qu’il retournait à Dieu, se leva de table, ôta son manteau et, ayant pris un linge, il s’en ceignit. Puis il versa de l’eau dans un bassin, il se mit à laver les pieds de ses disciples et à les essuyer avec le linge qu’il avait attaché autour de lui. Il vint donc à Simon-Pierre. Mais Pierre lui dit : « Quoi, Seigneur, vous me laveriez les pieds ! » Jésus lui répondit : « Vous ne comprenez pas maintenant ce que je fais, mais vous le saurez bientôt. » Pierre lui dit : « Jamais vous ne me laverez les pieds. » Jésus lui répondit : « Si je ne te lave, tu n’auras point de part avec moi. » Simon Pierre lui dit : « Seigneur, non seulement les pieds, mais aussi les mains et la tête. » Jésus lui dit : « Celui que le bain a déjà purifié n’a besoin que de se laver les pieds ; il est pur dans tout son corps ; pour vous, vous êtes purs, mais non pas tous. » Il connaissait celui qui le devait trahir, c’est pourquoi il dit : Vous n’êtes pas tous purs. Après donc qu’il leur eut lavé les pieds et qu’il eut repris son manteau, il se remit à table et leur dit : « Savez-vous ce que je viens de faire ? Vous m’appelez Maître et Seigneur, et vous dites bien, car je le suis. Si donc je vous ai lavé les pieds, moi, le Seigneur et le Maître, vous devez, vous aussi vous laver les pieds les uns aux autres. Car je vous ai donné l’exemple, afin que, comme je vous ai fait, vous fassiez aussi.