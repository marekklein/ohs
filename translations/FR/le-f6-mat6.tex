%\versseparator
Ils ont aiguisé leurs langues comme un glaive. Que les Juifs ne disent pas: Nous n’avons pas tué le Christ. Il est vrai qu’ils le mirent entre les mains du juge Pilate, afin de paraître en quelque sorte innocents de sa mort. Car Pilate leur avant dit : Faites-le mourir vous-mêmes ; ils répondirent : Il ne nous est pas permis de faire mourir quelqu’un. Ils voulaient par là rejeter l’injustice de leur forfait sur la personne du juge; mais pouvaient-ils tromper Dieu qui est juge aussi. Il est vrai que le procédé de Pilate l’a rendu participant de leur crime ; mais si on le compare à eux, on le trouve beaucoup moins criminel. Car il fit tout ce qu’il put pour le tirer de leurs mains ; et ce fut pour cela qu’il le leur montra tout déchiré de coups de fouet. Il fit flageller le Seigneur, non à dessein de le perdre , mais pour donner quelque chose à leur fureur; afin que du moins la vue de l’état dans lequel l’avait mis la flagellation pût les adoucir , et qu’ils cessassent de demander sa mort ; voilà ce qu’il fit. Mais voyant qu’ils persévéraient dans leur poursuite, vous savez qu’il lava ses mains, et qu’il leur dit que ce n’était pas lui qui était l’auteur de la mort de Jésus, et qu’il en était innocent. Il le fit mourir néanmoins. Mais s’il est coupable pour l’avoir condamné malgré lui, sont-ils innocents, ceux qui lui firent violence pour obtenir cette condamnation ? Non, sans doute. Pilate, en rendant sa sentence, et en ordonnant qu’il fût crucifié, l’a comme immolé crucifié lui-même. Mais c’est vous, ô Juifs, qui l’avez réellement immolé. Et comment? Par le glaive de votre langue ; car vous avez aiguisé vos langues comme l’épée. Et quand l’avez-vous frappé, si ce n’est au moment où vous poussâtes ce cri : Crucifiez-le, crucifiez-le?