%\versseparator
Blížili sa sviatky Nekvasených chlebov, ktoré sa nazývajú Veľká noc. Veľkňazi a zákonníci hľadali spôsob, ako ho zabiť; báli sa však ľudu. Tu vošiel satan do Judáša, ktorý sa volal Iškariotský a bol jedným z Dvanástich; i šiel a dohovoril sa s veľkňazmi a veliteľmi stráže, ako im ho vydá. Tí sa potešili a dohodli sa, že mu dajú peniaze. On súhlasil a hľadal príležitosť vydať im ho, keď s ním nebude zástup.
\versseparator
Prišiel deň Nekvasených chlebov, keď bolo treba zabiť veľkonočného baránka. Poslal Petra a Jána so slovami: „Choďte a pripravte nám veľkonočnú večeru!“ Oni sa ho opýtali: „Kde ju máme pripraviť?“ Povedal im: „Len čo vojdete do mesta, stretnete človeka, ktorý bude niesť džbán vody. Choďte za ním do domu, do ktorého vojde, a majiteľovi domu povedzte: »Učiteľ ti odkazuje: Kde je miestnosť, v ktorej by som mohol jesť so svojimi učeníkmi veľkonočného baránka?« On vám ukáže veľkú prestretú hornú sieň. Tam pripravte.“ Išli teda a všetko našli tak, ako im povedal. A pripravili veľkonočného baránka.
\versseparator
Keď prišla hodina, zasadol za stôl a apoštoli s ním. Tu im povedal: „Veľmi som túžil jesť s vami tohto veľkonočného baránka skôr, ako budem trpieť. Lebo hovorím vám: Už ho nebudem jesť, kým sa nenaplní v Božom kráľovstve.“ Vzal kalich, vzdával vďaky a povedal: „Vezmite ho a rozdeľte si ho medzi sebou. Lebo hovorím vám: Odteraz už nebudem piť z plodu viniča, kým nepríde Božie kráľovstvo.“ 
Potom vzal chlieb a vzdával vďaky, lámal ho a dával im, hovoriac: „Toto je moje telo, ktoré sa dáva za vás. Toto robte na moju pamiatku.“ 
Podobne po večeri vzal kalich a hovoril: „Tento kalich je nová zmluva v mojej krvi, ktorá sa vylieva za vás.
\versseparator
A hľa, ruka môjho zradcu je so mnou na stole. Syn človeka síce ide, ako je určené, ale beda človeku, ktorý ho zrádza!“ A oni sa začali jeden druhého vypytovať, kto z nich by to mohol urobiť.
\versseparator
Vznikol medzi nimi aj spor, kto je z nich asi najväčší. 
Povedal im: „Králi národov panujú nad nimi, a tí, čo majú nad nimi moc, volajú sa dobrodincami. Ale vy nie tak! Kto je medzi vami najväčší, nech je ako najmenší a vodca ako služobník. Veď kto je väčší? Ten, čo sedí za stolom, či ten, čo obsluhuje? Nie ten, čo sedí za stolom? A ja som medzi vami ako ten, čo obsluhuje.
\versseparator
Vy ste vytrvali so mnou v mojich skúškach a ja vám dám kráľovstvo, ako ho môj Otec dal mne, aby ste jedli a pili pri mojom stole v mojom kráľovstve, sedeli na trónoch a súdili dvanásť kmeňov Izraela.
\versseparator
Šimon, Šimon, hľa, satan si vás vyžiadal, aby vás preosial ako pšenicu. Ale ja som prosil za teba, aby neochabla tvoja viera. A ty, až sa raz obrátiš, posilňuj svojich bratov.“ On mu povedal: „Pane, hotový som ísť s tebou do väzenia i na smrť.“ Ale Ježiš povedal: „Hovorím ti, Peter, dnes nezaspieva kohút, kým tri razy nezaprieš, že ma poznáš.“
\versseparator
Potom im povedal: „Chýbalo vám niečo, keď som vás poslal bez mešca, bez kapsy a bez obuvi?“ Oni odpovedali: „Nie.“ On im povedal: „Ale teraz, kto má mešec, nech si ho vezme, takisto aj kapsu, a kto nemá, nech predá šaty a kúpi si meč. 
Lebo hovorím vám: Musí sa na mne splniť, čo je napísané: »Započítali ho medzi zločincov.« Lebo sa spĺňa o mne všetko.“ 
Oni hovorili: „Pane, pozri, tu sú dva meče.“ On im povedal: „Stačí.“
\versseparator
Potom vyšiel von a ako zvyčajne šiel na Olivovú horu a učeníci išli za ním. Keď prišiel na miesto, povedal im: „Modlite sa, aby ste neprišli do pokušenia!“ Sám sa od nich vzdialil asi toľko, čo by kameňom dohodil, kľakol si a modlil sa: „Otče, ak chceš, vezmi odo mňa tento kalich! No nie moja, ale tvoja vôľa nech sa stane!“ Tu sa mu zjavil anjel z neba a posilňoval ho. A on sa v smrteľnej úzkosti ešte vrúcnejšie modlil, pričom mu pot stekal na zem ako kvapky krvi. Keď vstal od modlitby a vrátil sa k učeníkom, našiel ich spať od zármutku. I povedal im: „Čo spíte? Vstaňte, modlite sa, aby ste neprišli do pokušenia!“
\versseparator
Kým ešte hovoril, zjavil sa zástup a pred nimi išiel jeden z Dvanástich, ktorý sa volal Judáš. Priblížil sa k Ježišovi, aby ho pobozkal. Ježiš mu však povedal: „Judáš, bozkom zrádzaš Syna človeka?“ Keď tí, čo boli okolo neho, videli, čo sa chystá, povedali: „Pane, máme udrieť mečom?“ A jeden z nich zasiahol veľkňazovho sluhu a odťal mu pravé ucho. Ale Ježiš povedal: „Nechajte to už!“ I dotkol sa mu ucha a uzdravil ho. Potom Ježiš povedal veľkňazom, veliteľom chrámovej stráže a starším, čo prišli za ním: „Vyšli ste s mečmi a kyjmi ako na zločinca. Keď som bol deň čo deň s vami v chráme, nepoložili ste na mňa ruky. Ale toto je vaša hodina a moc temna.“
\versseparator
Potom ho zajali, odviedli a zaviedli do veľkňazovho domu. Peter šiel zďaleka za nimi. Keď uprostred nádvoria rozložili oheň a posadali si okolo neho, Peter si sadol medzi nich. Ako tak sedel pri svetle, všimla si ho ktorási slúžka, zahľadela sa naňho a povedala: „Aj tento bol s ním.“ Ale on ho zaprel: „Žena, nepoznám ho.“ O chvíľu si ho všimol iný a povedal: „Aj ty si z nich.“ Peter povedal: „Človeče, nie som.“ A keď prešla asi hodina, ktosi iný tvrdil: „Veru, aj tento bol s ním, veď je aj Galilejčan.“ Peter povedal: „Človeče, neviem, čo hovoríš.“ A vtom, kým ešte hovoril, zaspieval kohút. 
Vtedy sa Pán obrátil a pozrel sa na Petra a Peter sa rozpamätal na Pánovo slovo, ako mu povedal: „Skôr ako dnes kohút zaspieva, tri razy ma zaprieš.“ Vyšiel von a horko sa rozplakal.
\versseparator
Muži, ktorí Ježiša strážili, posmievali sa mu a bili ho. Zakryli ho a vypytovali sa ho: „Prorokuj, hádaj, kto ťa udrel!“ A ešte všelijako ináč sa mu rúhali.
Keď sa rozodnilo, zišli sa starší ľudu, veľkňazi a zákonníci, predviedli ho pred svoju radu a hovorili mu: „Ak si Mesiáš, povedz nám to!“ On im odvetil: „Aj keď vám to poviem, neuveríte, a keď sa opýtam, neodpoviete mi. 
Ale odteraz bude Syn človeka sedieť po pravici Božej moci.“ Tu povedali všetci: „Si teda Boží Syn?“ On im povedal: „Vy sami hovoríte, že som.“ Oni povedali: „Načo ešte potrebujeme svedectvo? Veď sme to sami počuli z jeho úst!“
\versseparator
Tu celé zhromaždenie vstalo, odviedli ho k Pilátovi a začali naňho žalovať: „Tohto sme pristihli, ako rozvracia náš národ, zakazuje platiť dane cisárovi a tvrdí o sebe, že je Mesiáš, kráľ.“ Pilát sa ho spýtal: „Si židovský kráľ?“ On odpovedal: „Sám to hovoríš.“ Pilát povedal veľkňazom a zástupom: „Ja nenachádzam nijakú vinu na tomto človeku.“ Ale oni naliehali: „Poburuje ľud a učí po celej Judei; počnúc od Galiley až sem.“
\versseparator
Len čo to Pilát počul, opýtal sa, či je ten človek Galilejčan. 
A keď sa dozvedel, že podlieha Herodesovej právomoci, poslal ho k Herodesovi, lebo aj on bol v tých dňoch v Jeruzaleme. Keď Herodes uvidel Ježiša, veľmi sa zaradoval. Už dávno ho túžil vidieť, lebo o ňom počul, a dúfal, že ho uvidí urobiť nejaký zázrak. Mnoho sa ho vypytoval, ale on mu na nič neodpovedal. Stáli tam aj veľkňazi a zákonníci a nástojčivo naň žalovali. Ale Herodes so svojimi vojakmi ním opovrhol, urobil si z neho posmech, dal ho obliecť do bielych šiat a poslal ho nazad k Pilátovi. V ten deň sa Herodes a Pilát spriatelili, lebo predtým žili v nepriateľstve.
\versseparator
Pilát zvolal veľkňazov, predstavených a ľud a povedal im: „Priviedli ste mi tohto človeka, že poburuje ľud. Ja som ho pred vami vypočúval a nenašiel som na tomto človeku nič z toho, čo na neho žalujete. Ale ani Herodes, lebo nám ho poslal späť. Vidíte, že neurobil nič, za čo by si zasluhoval smrť. Potrescem ho teda a prepustím.“
Tu celý dav skríkol: „Preč s ním a prepusť nám Barabáša!“ Ten bol uväznený pre akúsi vzburu v meste a pre vraždu.
Pilát k nim znova prehovoril, lebo chcel Ježiša prepustiť. Ale oni vykrikovali: „Ukrižuj! Ukrižuj ho!“ On k nim tretí raz prehovoril: „A čo zlé urobil? Nenašiel som na ňom nič, za čo by si zasluhoval smrť. Potrestám ho teda a prepustím.“ Ale oni veľkým krikom dorážali a žiadali, aby ho dal ukrižovať. Ich krik sa stupňoval a Pilát sa rozhodol vyhovieť ich žiadosti: prepustil toho, ktorého si žiadali, čo bol uväznený pre vzburu a vraždu, kým Ježiša vydal ich zvoli.
\versseparator
Ako ho viedli, chytili istého Šimona z Cyrény, ktorý sa vracal z poľa, a položili naň kríž, aby ho niesol za Ježišom.
Šiel za ním veľký zástup ľudu aj žien, ktoré nad ním kvílili a nariekali. Ježiš sa k nim obrátil a povedal: „Dcéry jeruzalemské, neplačte nado mnou, ale plačte samy nad sebou a nad svojimi deťmi. Lebo prichádzajú dni, keď povedia: »Blahoslavené neplodné, loná, čo nerodili, a prsia, čo nepridájali!«
Vtedy začnú hovoriť vrchom: »Padnite na nás!« a kopcom: »Prikryte nás!« 
Lebo keď toto robia so zeleným stromom, čo sa stane so suchým?“
Vedno s ním viedli na popravu ešte dvoch zločincov.
\versseparator
Keď prišli na miesto, ktoré sa volá Lebka, ukrižovali jeho i zločincov: jedného sprava, druhého zľava. Ježiš povedal: „Otče, odpusť im, lebo nevedia, čo robia.“
Potom hodili lós a rozdelili si jeho šaty. 
Ľud tam stál a díval sa. Poprední muži sa mu posmievali a vraveli: „Iných zachraňoval, nech zachráni aj seba, ak je Boží Mesiáš, ten vyvolenec.“ Aj vojaci sa mu posmievali. Chodili k nemu, podávali mu ocot a hovorili: „Zachráň sa, ak si židovský kráľ!“ Nad ním bol nápis: „Toto je židovský kráľ.“
\versseparator
A jeden zo zločincov, čo viseli na kríži, sa mu rúhal: „Nie si ty Mesiáš?! Zachráň seba i nás!“ Ale druhý ho zahriakol: „Ani ty sa nebojíš Boha, hoci si odsúdený na to isté? Lenže my spravodlivo, lebo dostávame, čo sme si skutkami zaslúžili. Ale on neurobil nič zlé.“ Potom povedal: „Ježišu, spomeň si na mňa, keď prídeš do svojho kráľovstva.“ On mu odpovedal: „Veru, hovorím ti: Dnes budeš so mnou v raji.“
\versseparator
Bolo už okolo dvanástej hodiny a nastala tma po celej zemi až do tretej hodiny popoludní. Slnko sa zatmelo, chrámová opona sa roztrhla napoly a Ježiš zvolal mocným hlasom: „Otče, do tvojich rúk porúčam svojho ducha.“ Po tých slovách vydýchol.
\versseparator
Keď stotník videl, čo sa stalo, oslavoval Boha, hovoriac: „Tento človek bol naozaj spravodlivý.“ A celé zástupy tých, čo sa zišli na toto divadlo a videli, čo sa deje, bili sa do pŕs a vracali sa domov.
Všetci jeho známi stáli obďaleč i ženy, ktoré ho sprevádzali z Galiley, a dívali sa na to.
\versseparator
Tu istý muž menom Jozef, člen rady, dobrý a spravodlivý človek z judejského mesta Arimatey, ktorý nesúhlasil s ich rozhodnutím ani činmi a očakával Božie kráľovstvo, zašiel k Pilátovi a poprosil o Ježišovo telo. Keď ho sňal, zavinul ho do plátna a uložil do vytesaného hrobu, v ktorom ešte nik neležal.