Ex tractátu sancti Augustíni Epíscopi super Psalmos
Exáudi, Deus, oratiónem meam, et ne despéxeris deprecatiónem meam:
inténde mihi, et exáudi me. Satagéntis, solliciti, in tribulatióne pósiti,
verba sunt ista. Orat multa pátiens, de malo liberari desiderans. Supe-
rest ut videámus in quo malo sit: et cum dicere cœperit, agnoscámus
ibi nos esse: ut communicáta tribulatióne, conjungámus oratiónem.
Contristátus sum, inquit, in exercitatióne mea, et conturbátus sum.
Ubi contristátus? ubi conturbátus? In exercitatióne mea, inquit. Hó-
mines malos, quos pátitur, commemorátus est: eamdemque passió-
nem malórum hóminum exercitatiónem suam dixit. Ne putetis gratis
esse malos in hoc mundo, et nihil boni de illis ágere Deum. Omnis
malus aut ideo vivit, ut corrigátur; aut ideo vivit, ut per illum bonus
exerceátur.

Give ear to my prayer, O God, and despise not my supplication: 
attend unto me and hear me. These are the words of a man travailing, anxious, and troubled. 
He prayeth in the midst of much suffering, longing to be rid of his affliction. 
Our part is to see what that his affliction was, and when he hath told us, 
to acknowledge that we also suffer therefrom; 
that so, partaking in his trouble, we may take part also in his exercise, 
and am troubled. Wherein mourned he? Wherein was he troubled? He saith: 
In my exercise. In the next words he giveth us to know that his affliction was the oppression of the wicked, 
because of the voice of the enemy, and because of the oppression of the wicked, and this suffering which came upon him at the hands of wicked men, he hath called his exercise. Think not that wicked men are in this world for nothing, or that God doth no good with them. Every wicked man liveth, either to repent, or to exercise the righteous.