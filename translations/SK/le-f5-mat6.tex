"Lebo v meste vidím neprávosť a hádky," (verš 10)
Všimnime si slávu samotného kríža.
Ku krížu, ktorý bol predmetom opvrhnutia nepriateľov Boha, k tomuto krížu teraz vzhliadajú králi. Dokázal svoju silu - podmanil si svet, nie železom, ale drevom. Nepriatelia Boha si mysleli, že je to pre nich objekt opovrhnutia a zábavy - stáli pred ním a hovorili: ak si boží syn, zostúp z kríža (Mat. XXVII-39,40). Vystieral teda ruky k ľudu nevernému a vzdorovitému (Rim. X, 21). Ak spravodlivý žije z viery (Rim I,17), kto nemá vieru, nie je spravodlivý. Preto ak je napísané "neprávosť", môžeme rozumieť "neviera". Ak teda Pán videl v meste neprávosť a hádky, a vystieral ruky k ľudu nevernému a vzdorovitému, zatúžil po ich spasení a povedal: Otče, odpusť im, lebo nevedia, čo činia (Lukáš XXIII, 34). 