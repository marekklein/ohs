Vau. 
Sionskú dcéru opustila
všetka jej nádhera,
jej kniežatá sú sťa jelene,
čo pastvu nenachodia,
čo bezvládne sa vlečú
pred tvárou stíhača.
\versseparator
Zain.
Jeruzalem spomína na dni
plaču a biedy,
(na všetky svoje skvosty,
ktoré mal od dávnych dní);
keď mu ľud padal rukou nepriateľa
a nemal pomocníka,
nepriateľ sa naň díval, smial sa
jeho záhube.
\versseparator
Chet.
Pochybil Jeruzalem veľmi,
preto sa stal odporným,
opovrhli ním všetci ctitelia,
keď uzreli jeho hanbu,
on sám však vzdychá,
odvracia sa preč.
\versseparator
Tet.
Škvrny sú na jeho vlečkách,
na následky si nespomenul
a poklesol úžasne,
nemá tešiteľa.
„Na moju biedu pozri, Pane,
nepriateľ ma, hľa, premohol!“
\versseparator
Jeruzalem, navráť sa k Pánovi, svojmu Bohu.