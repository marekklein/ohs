%\versseparator
Bolo dva dni pred Veľkou nocou a sviatkami Nekvasených chlebov. Veľkňazi a zákonníci hľadali spôsob, ako ho podvodne chytiť a zabiť. Ale hovorili: „Nie vo sviatok, aby sa ľud nevzbúril.“
\versseparator
Keď bol v Betánii v dome Šimona Malomocného a sedel pri stole, prišla žena s alabastrovou nádobou pravého vzácneho nardového oleja. Nádobu rozbila a olej mu vyliala na hlavu. Niektorí sa hnevali a hovorili si: „Načo takto mrhať voňavý olej?! Veď sa mohol tento olej predať za viac ako tristo denárov a tie rozdať chudobným.“ A osopovali sa na ňu.
Ale Ježiš povedal: „Nechajte ju! Prečo ju trápite? Urobila mi dobrý skutok. Veď chudobných máte vždy medzi sebou, a keď budete chcieť, môžete im robiť dobre. Ale mňa nemáte vždy.
Urobila, čo mohla. Vopred pomazala moje telo na pohreb. Veru, hovorím vám: Kdekoľvek na svete sa bude ohlasovať evanjelium, bude sa na jej pamiatku hovoriť aj o tom, čo urobila.“
\versseparator
Judáš Iškariotský, jeden z Dvanástich, odišiel k veľkňazom, aby im ho zradil. Tí sa potešili, keď to počuli, a sľúbili, že mu dajú peniaze. A on hľadal spôsob, ako ho príhodne vydať.
\versseparator
V prvý deň sviatkov Nekvasených chlebov, keď zabíjali veľkonočného baránka, povedali mu jeho učeníci: „Kde ti máme ísť pripraviť veľkonočnú večeru?“ Poslal dvoch zo svojich učeníkov a vravel im: „Choďte do mesta. Tam stretnete človeka, ktorý bude niesť džbán vody. Choďte za ním
a pánovi domu, do ktorého vojde, povedzte: »Učiteľ odkazuje: Kde je pre mňa miestnosť, v ktorej by som mohol jesť so svojimi učeníkmi veľkonočného baránka?« On vám ukáže veľkú hornú sieň, prestretú a pripravenú. Tam nám prichystajte.“ Učeníci odišli, a keď prišli do mesta, všetko našli tak, ako im povedal. A pripravili veľkonočného baránka.
\versseparator
Keď sa zvečerilo, prišiel s Dvanástimi.
A keď boli pri stole a jedli, Ježiš povedal: „Veru, hovorím vám: Jeden z vás ma zradí, ten, čo je so mnou.“ Zosmutneli a začali sa ho jeden po druhom vypytovať: „Azda ja?“ On im odpovedal: „Jeden z Dvanástich, čo so mnou namáča v mise. Syn človeka síce ide, ako je o ňom napísané, ale beda človeku, ktorý zrádza Syna človeka! Pre toho človeka by bolo lepšie, keby sa nebol narodil.“
\versseparator
Keď jedli, vzal chlieb a dobrorečil, lámal ho a dával im, hovoriac: „Vezmite, toto je moje telo!“ Potom vzal kalich, vzdával vďaky, dal im ho a všetci z neho pili. A povedal im: „Toto je moja krv novej zmluvy, ktorá sa vylieva za všetkých. Veru, hovorím vám: Už nebudem piť z plodu viniča až do dňa, keď ho budem piť nový v Božom kráľovstve.“
\versseparator
Potom zaspievali chválospev a vyšli na Olivovú horu.
Vtedy im Ježiš povedal: „Všetci odpadnete, lebo je napísané: »Udriem pastiera a ovce sa rozpŕchnu.« 
Ale keď vstanem z mŕtvych, predídem vás do Galiley.“ Peter mu povedal: „Aj keby všetci odpadli, ja nie.“ Ježiš mu odvetil: „Veru, hovorím ti: Ty ma dnes, tejto noci, skôr, ako dva razy kohút zaspieva, tri razy zaprieš.“ Ale on ešte horlivejšie vyhlasoval: „Aj keby som mal umrieť s tebou, nezapriem ťa.“ Podobne hovorili aj ostatní.
\versseparator
Prišli na pozemok, ktorý sa volá Getsemani, a povedal svojim učeníkom: „Sadnite si tu, kým sa pomodlím.“ Vzal so sebou Petra, Jakuba a Jána. I doľahla naňho hrôza a úzkosť. Vtedy im povedal: „Moja duša je smutná až na smrť. Ostaňte tu a bdejte!“ Trocha poodišiel, padol na zem a modlil sa, aby ho, ak je možné, minula táto hodina.
Hovoril: „Abba, Otče! Tebe je všetko možné. Vezmi odo mňa tento kalich. No nie čo ja chcem, ale čo ty.“ Keď sa vrátil, našiel ich spať. I povedal Petrovi: „Šimon, spíš? Ani hodinu si nemohol bdieť? Bdejte a modlite sa, aby ste neprišli do pokušenia. Duch je síce ochotný, ale telo slabé.“ Znova odišiel a modlil sa tými istými slovami. A keď sa vrátil, zasa ich našiel spať: oči sa im zatvárali od únavy a nevedeli, čo mu povedať. Keď prišiel tretí raz, povedal im: „Ešte spíte a odpočívate? Dosť už. Prišla hodina: hľa, Syna človeka už vydávajú do rúk hriešnikov. Vstaňte, poďme! Pozrite, môj zradca je blízko.“
\versseparator
A kým ešte hovoril, prišiel zrazu Judáš, jeden z Dvanástich, a s ním zástup s mečmi a kyjmi, ktorý poslali veľkňazi, zákonníci a starší. 
Jeho zradca im dal znamenie: „Koho pobozkám, to je on. Chyťte ho a obozretne odveďte!“ Keď prišiel, hneď pristúpil k nemu a povedal: „Rabbi.“ A pobozkal ho. Oni položili naň ruky a zajali ho. Tu jeden z okolostojacich vytasil meč, zasiahol ním veľkňazovho sluhu a odťal mu ucho. Ježiš im povedal: „Vyšli ste s mečmi a kyjmi ako na zločinca, aby ste ma zajali. Deň čo deň som učil u vás v chráme, a nezajali ste ma. Ale musí sa splniť Písmo.“ Vtedy ho všetci opustili a rozutekali sa.
No akýsi mladík išiel za ním, odetý plachtou na holom tele; a chytili ho. Ale on pustil plachtu a utiekol nahý.
\versseparator
Ježiša priviedli k veľkňazovi, kde sa zhromaždili všetci veľkňazi, starší a zákonníci. Peter šiel zďaleka za ním až dnu do veľkňazovho dvora. Sadol si k sluhom a zohrieval sa pri ohni. Veľkňazi a celá veľrada zháňali svedectvo proti Ježišovi, aby ho mohli odsúdiť na smrť. Ale nenašli. Mnohí proti nemu krivo svedčili, a ich svedectvá sa nezhodovali. Tu niektorí vstali a krivo proti nemu svedčili: „My sme ho počuli hovoriť: »Ja zborím tento chrám zhotovený rukou a za tri dni postavím iný, nie rukou zhotovený.«“ Ale ani tak sa ich svedectvo nezhodovalo. Tu vstal veľkňaz, postavil sa do stredu a opýtal sa Ježiša: „Nič neodpovieš na to, čo títo svedčia proti tebe?“
Ale on mlčal a nič neodpovedal. Veľkňaz sa ho znova pýtal: „Si ty Mesiáš, syn Požehnaného?“ 
Ježiš odvetil: „Áno, som. A uvidíte Syna človeka sedieť po pravici Moci a prichádzať s nebeskými oblakmi.“
Vtedy si veľkňaz roztrhol rúcho a povedal: „Načo ešte potrebujeme svedkov? Počuli ste rúhanie. Čo na to poviete?“ A oni všetci vyniesli nad ním súd, že je hoden smrti.
Niektorí začali naňho pľuť, zakrývali mu tvár, bili ho päsťami a hovorili mu: „Prorokuj!“ Aj sluhovia ho bili po tvári.
\versseparator
Keď bol Peter dolu na nádvorí, prišla jedna z veľkňazových slúžok. Len čo zbadala Petra, ako sa zohrieva, pozrela sa naňho a povedala: „Aj ty si bol s tým Nazaretčanom, Ježišom.“ 
Ale on zaprel: „Ani neviem, ani nerozumiem, čo hovoríš.“ Vyšiel von pred nádvorie a zaspieval kohút. Keď ho tam videla slúžka, znova začala vravieť okolostojacim: „Tento je z nich.“ Ale on opäť zapieral. O chvíľku tí, čo tam stáli, znova hovorili Petrovi: „Veru si z nich, veď si aj Galilejčan.“ On sa však začal zaklínať a prisahať: „Nepoznám toho človeka, o ktorom hovoríte.“ Vtom kohút zaspieval druhý raz. Vtedy sa Peter rozpamätal na slovo, ktoré mu bol povedal Ježiš: „Skôr ako dva razy kohút zaspieva, tri razy ma zaprieš.“ I rozplakal sa.
\versseparator
Hneď zrána mali poradu veľkňazi so staršími a zákonníkmi, teda celá veľrada. Ježiša spútali, odviedli a odovzdali Pilátovi. Pilát sa ho spýtal: „Si židovský kráľ?“ On mu odpovedal: „Sám to hovoríš.“ Veľkňazi naň mnoho žalovali a Pilát sa ho znova spytoval: „Nič neodpovieš? Pozri, čo všetko žalujú na teba!“ Ale Ježiš už nič nepovedal, takže sa Pilát čudoval.
\versseparator
Na sviatky im prepúšťal jedného väzňa, ktorého si žiadali. S povstalcami, čo sa pri vzbure dopustili vraždy, bol uväznený muž, ktorý sa volal Barabáš. Zástup vystúpil hore a žiadal si to, čo im robieval. Pilát im povedal: „Chcete, aby som vám prepustil židovského kráľa?“ Lebo vedel, že ho veľkňazi vydali zo závisti. Ale veľkňazi podnietili zástup, aby im radšej prepustil Barabáša. Pilát sa ich znova opýtal: „Čo mám teda podľa vás urobiť so židovským kráľom?“ Oni opäť skríkli: „Ukrižuj ho!“ Pilát im vravel: „A čo zlé urobil?“ Ale oni tým väčšmi kričali: „Ukrižuj ho!“ A Pilát, aby urobil ľudu po vôli, prepustil im Barabáša. Ježiša však dal zbičovať a vydal ho, aby ho ukrižovali.
\versseparator
Vojaci ho odviedli dnu do nádvoria, čiže do vládnej budovy, a zvolali celú kohortu. Odeli ho do purpurového plášťa, z tŕnia uplietli korunu a založili mu ju a začali ho pozdravovať: „Buď pozdravený, židovský kráľ!“ Bili ho trstinou po hlave, pľuli naňho, kľakali pred ním a klaňali sa mu.
Keď sa mu naposmievali, vyzliekli ho z purpuru a obliekli mu jeho šaty. Potom ho vyviedli, aby ho ukrižovali.
\versseparator
Tu prinútili istého Šimona z Cyrény, Alexandrovho a Rúfovho otca, ktorý sa tade vracal z poľa, aby mu niesol kríž. Tak ho priviedli na miesto Golgota, čo v preklade znamená Lebka.
Dávali mu víno zmiešané s myrhou, ale on ho neprijal.
Potom ho ukrižovali a rozdelili si jeho šaty – hodili o ne lós, kto si má čo vziať.
\versseparator
Keď ho ukrižovali, bolo deväť hodín. Jeho vinu označili nápisom: „Židovský kráľ.“ Vedno s ním ukrižovali aj dvoch zločincov: jedného napravo od neho, druhého naľavo.
\versseparator
A tí, čo šli okolo, rúhali sa mu; potriasali hlavami a vraveli: „Aha, ten, čo zborí chrám a za tri dni ho postaví. Zachráň sám seba, zostúp z kríža!“ Podobne sa mu posmievali aj veľkňazi a so zákonníkmi si hovorili: „Iných zachraňoval, sám seba nemôže zachrániť. Kristus, kráľ Izraela! Nech teraz zostúpi z kríža, aby sme videli a uverili.“ Ešte aj tí ho hanobili, čo boli s ním ukrižovaní.
\versseparator
Keď bolo dvanásť hodín, nastala tma po celej zemi až do tretej hodiny popoludní. O tretej hodine zvolal Ježiš mocným hlasom: „Heloi, heloi, lema sabakthani?“, čo v preklade znamená: „Bože môj, Bože môj, prečo si ma opustil?“ Keď to počuli, niektorí z okolostojacich vraveli: „Pozrite, volá Eliáša.“
Ktosi odbehol, naplnil špongiu octom, nastokol ju na trstinu, dával mu piť a hovoril: „Počkajte, uvidíme, či ho Eliáš príde sňať.“ Ale Ježiš zvolal mocným hlasom a vydýchol.
\versseparator
Chrámová opona sa roztrhla vo dvoje odvrchu až dospodku. Keď stotník, čo stál naproti nemu, videl, ako vykríkol a skonal, povedal: „Tento človek bol naozaj Boží Syn.“
\versseparator
Zobďaleč sa pozerali aj ženy. Medzi nimi Mária Magdaléna, Mária, matka Jakuba Mladšieho a Jozesa, i Salome, ktoré ho sprevádzali a posluhovali mu, keď bol v Galilei. A mnohé iné, čo s ním prišli do Jeruzalema.
Keď sa už zvečerilo, pretože bol Prípravný deň, čiže deň pred sobotou, prišiel Jozef z Arimatey, významný člen rady, ktorý tiež očakával Božie kráľovstvo, smelo vošiel k Pilátovi a poprosil o Ježišovo telo.
Pilát sa zadivil, že už zomrel. Zavolal si stotníka a opýtal sa ho, či je už mŕtvy. Keď mu to stotník potvrdil, daroval telo Jozefovi. On kúpil plátno, a keď ho sňal, zavinul ho do plátna a uložil do hrobu vytesaného do skaly. A ku vchodu do hrobu privalil kameň. 