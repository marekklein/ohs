Ospevujme veľký súboj,
šírme slávne odkazy
trámov kríža, ktorý zdolal
strojcu večnej nákazy!
Obeť na ňom zmiera za svet
a tou smrťou víťazí.
\versseparator
Prvý človek veril zvodom,
smrť si vtiahol do domu,
keď v ňom Božiu lásku zhasil
jed, čo zjedol zo stromu.
Lež Boh nový strom si vybral
zažehnať tú pohromu.
\versseparator
Poriadok si vyžadoval
po bolestnom otrase
do priepasti vrhnúť zhubcu,
čo nás ranil v zápase,
aby skazonosný úder
poslúžil nám ku spáse.
\versseparator
Keď čas prišiel, keď po Bohu
vyvrcholil dávny smäd,
Otec poslal na svet Syna,
skrz ktorého stvoril svet,
a on z Panny ľudsky vzklíčil
sťa z kra vytúžený kvet.
\versseparator
A keď sa mu v ľudskom veku
najkrajší vek priblíži,
predurčený trpieť, rád sa
pred mukami poníži,
preto Baránka, hľa, ľudstvo
v obeť dvíha na kríži.
\versseparator
Tebe, Otče, s Duchom Svätým
nech je sláva naveky
skrze Krista, ktorý telom
nie je nám už ďaleký
a chce krížom zmeniť v radosť
ľudský bôľ a náreky. Amen.
