Z traktátu sv. Augustína, biskupa, o žalmoch.
Čuj, Bože, moju modlitbu
a pred mojou úpenlivou prosbou sa neskrývaj:
pohliadni na mňa a vyslyš ma.
Toto sú slová človeka ustarosteného, úzkostlivého a znepokojeného.
Modlí sa uprostred trápenia, túži aby sa skončilo.
Našou úlohou je prísť na to, čo bolo tým trápením a kedy nám o ňom povedal, aby sme uznali, že aj my tým trpíme.
Tým, že budeme mať účasť na jeho utrpení, môžeme mať účasť aj na jeho skúške.
Čím bol znepokojený? Z čoho bol smutný? Hovorí o svojej skúške.
V nasledujúcich slovách nám prezdrádza, že jeho utrpenie pochádza z útlaku hriešnika a nepriateľa. A tento útlak a utrpenie nazýva svojou skúškou.Nemyslime si, že zlo v tomto svete nemá žiaden význam. Každý zlý človek žije buď preto, aby sa napravil, alebo aby bol skúškou pre tých dobrých. 