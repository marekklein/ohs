\ohschapter{Ad Vesperas}
%\vspace{-4em}
%\nadpisd{cum psalmi iuxta vulgata nova}
\vspace{1em}

\redvbar Bože, príď mi na pomoc.

\redrbar Pane, ponáhľaj sa mi pomáhať. Sláva Otcu i Synu i Duchu Svätému. Ako bolo na počiatku, tak nech je i teraz i vždycky, i na veky vekov. Amen. Aleluja.
%\ohsformatverses
%\gresetinitiallines{1}%
%\grechangedim{annotationraise}{-2 mm}{scalable}
%\greannotation{\color{red} \Vbar}
%\gregorioscore{./other/deus-in-adjutorium-alleluia}
%%%%%%%%%%%%%%%%%%%%%%%%%%%%%%%%%%%%%%%%%%%%%%%%%%%%
\vspace{2em}

\nadpisb{Hymnus}
\hfill\begin{minipage}{\dimexpr\textwidth-2.4cm}
Ó, Mária bolestivá, naša ochrana,\newline 
slovenský náš národ volá: Pros za nás Boha.\newline 
[:Ty si Mať dobrotivá, Patrónka ľútostivá,\newline 
oroduj vždy za náš národ u svojho Syna.:]

\vspace{1em}

Matka drahá, spomni na nás tu pod Tatrami,\newline 
verní sme my svätej Cirkvi, prebývaj s nami!\newline 
[:Ty si Mať dobrotivá, Patrónka ľútostivá,\newline 
oroduj vždy za náš národ u svojho Syna.:]
\vspace{1em}

Po horách i po dolinách venčíme tvoj chrám,\newline 
pútne miesta volajú ti: Matka, národ chráň!\newline
[:Ty si Mať dobrotivá, Patrónka ľútostivá,\newline 
oroduj vždy za náš národ u svojho Syna.:]
%\xdef\tpd{\the\prevdepth}
\end{minipage}


%%%%%%%%%%%%%%%%%%%%%%%%%%%%%%%%%%%%%%%%%%%%%%%%%%%%
%\vspace{2em}
\pagebreak
\nadpisb{Psalmódia}
\ohsformatantiphona
\ohsannotation{Ant. 1}{II d}
\gregorioscore{./ant/ant-a_planta_pedis}
\ohstranslate{./ant-a_planta_pedis}
\ohspsalmus{Žalm 122}{./psalmivn/ps122sk}{./psalmivn/ps122sk.tex}
\ohsscorenostar{./ant/ant-a_planta_pedis}
\separator{3}

%%%%%%%%%%%%%%%%%%%%%%%%%%%%%%%%%%%%%%%%%%%%%%%%%%%%
\ohsformatantiphona
\ohsannotation{Ant. 2}{II d}
\gregorioscore{./ant/ant-oppresssit_me_dolor}
\ohstranslate{./ant-oppresssit_me_dolor}
\ohspsalmus{Žalm 127}{./psalmivn/ps127sk}{./psalmivn/ps127sk.tex}
\ohsscorenostar{./ant/ant-oppresssit_me_dolor}
\separator{4}

%%%%%%%%%%%%%%%%%%%%%%%%%%%%%%%%%%%%%%%%%%%%%%%%%%%%
\ohsformatantiphona
\ohsannotation{Ant. 3}{I a}
\gregorioscore{./ant/ant-dominus_dedit}
\ohstranslate{./ant-dominus_dedit}
\ohspsalmus{Chválospev Ef1, 3-10}{./psalmivn/ef1sk}{./psalmivn/ef1sk.tex}
\ohsscorenostar{./ant/ant-dominus_dedit}
\separator{3}

%%%%%%%%%%%%%%%%%%%%%%%%%%%%%%%%%%%%%%%%%%%%%%%%%%%%
\vspace{1em}
\nadpisb{Krátke čítanie}
\nadpisleft{Gal 4, 4-7} 

Keď prišla plnosť času, Boh poslal svojho Syna, narodeného zo ženy, narodeného pod zákonom, 
aby vykúpil tých, čo boli pod zákonom, a aby sme dostali adoptívne synovstvo. 
Pretože ste synmi, poslal Boh do našich sŕdc Ducha svojho Syna a on volá: „Abba, Otče!“ 
A tak už nie si otrok, ale syn; a keď syn, tak skrze Boha aj dedič.


%%%%%%%%%%%%%%%%%%%%%%%%%%%%%%%%%%%%%%%%%%%%%%%%%%%%
\pagebreak
\nadpisb{Meditácia (J. E. kardinál Pietro Parolin)}

\small{Drahí bratia a sestry,

1. Modlitba vešpier, ktorou sa lúčime s týmto dňom, chýliacim sa ku koncu, nás spojila v tejto krásnej a historickej katedrále Košíc, mesta, ktoré nás svojou geografickou polohou otvára svetu a kultúre Východu.
Srdečne pozdravujem J. E. Mons. Bernarda Bobera, Vášho arcibiskupa, J. E. Mons. Cyrila Vasiľa, košického arcibiskupa a eparchu (ak je prítomný), a všetkých prítomných biskupov, kňazov, rehoľníkov a rehoľníčky a všetkých vás, ktorí ste prítomní v tejto katedrále na modlitbe vešpier.

2. V krypte tejto katedrály je pochovaný Jeho Eminencia kardinál Jozef Tomko, ktorého si Pán povolal k sebe minulý rok a ktorého sté výročie narodenia budeme sláviť na budúci rok 11. marca. Kard. Tomko bol nielen veľkým služobníkom Svätého stolca, ktorý zastával najvyššie zodpovedné funkcie v univerzálnej Cirkvi, ale aj verným a príkladným svedkom slovenskej duchovnej a národnej identity. 

3. Počas ostatných Svetových dní mládeže v Lisabone pápež František zdôraznil dôležitosť kráčania a to, že všetci musíme kráčať s Pannou Máriou a zveriť sa jej materinskému pohľadu a ochrane. Spoločné kráčanie s Pannou Máriou nám dáva istotu a pomáha nám vytrvať vo viere a posilňovať ju. Mária vždy kráča (evanjeliá nám ju predstavujú ako Ženu na ceste) a my potrebujeme vieru, ktorá sa vždy vydáva na cestu... na putovanie spojené s hľadaním Pána.

4. Synodalita je práve výrazom Cirkvi na ceste a v Prípravnom dokumente na synodu (Instrumentum laboris) sú uvedené tri základné charakteristiky, ktoré túto cestu robia účinnou: spoločenstvo s vyznávaním našej viery, účasť, ktorá nás robí zodpovednými a aktívnymi, a misia, ktorá nás dáva do pohybu s evanjelizáciou zameranou na periférie sveta. Napredujeme ako veriaci ľud spolu s Máriou, ktorá nás vedie k Pánovi. Mária je Matkou cesty, ktorá sa vydáva na cestu (Homília pápeža Františka v Šaštíne 15. septembra 2021). Synoda, ktorú sa pripravujeme sláviť, od nás chce, aby sme si viac uvedomili, že všetci sme aktívnymi osobami, a nie pasívnymi subjektmi našich spoločenstiev, a že všetci pokrstení sa musia cítiť zapojení do dynamiky evanjelizácie a pastoračného života Cirkvi.

5. Preto, bratia a sestry, pokračujme v spoločnom kráčaní, lebo keď kráčame sami, riskujeme, že sa stratíme a vydáme sa po nesprávnej ceste. Keď kráčame spolu, máme predovšetkým možnosť stretnúť sa navzájom a toto stretnutie vytvára dôveru a vzájomnú úctu a umožňuje nám pomáhať si navzájom, robí nás odvážnymi, zbavuje nás všetkých predsudkov, otvára nás prijatiu každého bez toho, aby sme niekoho vylučovali, a necháva nás prikryť plášťom Panny Márie, ktorá nás drží pri sebe nablízku a chráni nás, pričom vždy udržiava nažive nádej.

6. Dnes večer sa modlíme modlitbu zverenia Panne Márii. K nej, Kráľovnej pokoja, sa prihovárame za ukončenie vojny na Ukrajine. Panna Mária vie, čo potrebujeme, a sme si istí, že sa o nás postará. Zverujeme jej všetky svoje potreby, obavy a nádeje. Zverujeme jej utrpenie tisícov našich bratov a sestier utečencov z Ukrajiny. Prosíme Presvätú Pannu, aby nám dodala entuziazmus a odvahu a pomohla nám kráčať vždy s ňou, aby sme prekonali všetky naše obavy a svedčili o kresťanských hodnotách.

Dnes večer, keď sa budeme vracať domov, zoberme Pannu Máriu do našich sŕdc a do našich domovov, aby ich chránila svojou materinskou láskou. Amen.}
%%%%%%%%%%%%%%%%%%%%%%%%%%%%%%%%%%%%%%%%%%%%%%%%%%%%
\vspace{1em}
\nadpisb{Krátke responzórium}
\redvbar Stála svätá Mária, Kráľovná neba a Pani sveta pri Pánovom kríži.

\redrbar Stála svätá Mária, Kráľovná neba a Pani sveta pri Pánovom kríži.

\redvbar Šťastná je, veď bez smrti získala palmu mučeníctva.

\redrbar Pri Pánovom kríži.

\redvbar Sláva Otcu i Synu i Duchu Svätému.

\redrbar Stála svätá Mária, Kráľovná neba a Pani sveta pri Pánovom kríži.

%%%%%%%%%%%%%%%%%%%%%%%%%%%%%%%%%%%%%%%%%%%%%%%%%%%%
%\vspace{1em}
\separator{4}
\nadpisb{Evanjeliový chválospev}

%%%%%%%%%%%%%%%%%%%%%%%%%%%%%%%%%%%%%%%%%%%%%%%%%%%%
%%%%%%%%%%%%%%%%%%%%%%%%%%%%%%%%%%%%%%%%%%%%%%%%%%%%
\ohsformatantiphona
\ohsannotation{Ant.}{VIII g}
\gregorioscore{./ant/ant-cum_vidisset}
\ohstranslate{./ant-cum_vidisset}
\ohspsalmus{Lk 1, 46-55}{./psalmivn/magnificat-sk}{./psalmivn/magnificat-sk.tex}
\ohsscorenostar{./ant/ant-cum_vidisset}
\separator{3}

%%%%%%%%%%%%%%%%%%%%%%%%%%%%%%%%%%%%%%
\nadpisb{Prosby}
Z celého srdca chváľme všemohúceho Boha Otca, lebo z jeho vôle oslavujú Pannu Máriu, matku jeho Syna, všetky pokolenia, a pokorne ho prosme:

\textit{Nech oroduje za nás Sedembolestná.}

Bože, ty konáš vo svete veľké veci: poníženú Pannu Máriu si ustanovil za matku Cirkvi; 
daj, nech ju blahoslavia všetky pokolenia.

\redrbar Nech oroduje za nás Sedembolestná.

Ty si nám dal Pannu Máriu za matku a patrónku;

na jej orodovanie udeľ nášmu národu pokoj a spásu.

\redrbar Nech oroduje za nás Sedembolestná.

Ty si posilňoval Pannu Máriu, keď stála pod krížom, a pri vzkriesení svojho syna si ju naplnil radosťou;

uľahči trpiacim a posilni ich nádej.

\redrbar Nech oroduje za nás Sedembolestná.

Ty si Pannu Máriu urobil pozornou voči tvojmu slovu a svojou vernou služobnicou;

na jej príhovor urob z našich biskupov a kňazov verných služobníkov tvojho Syna a správcov jeho tajomstiev.

\redrbar Nech oroduje za nás Sedembolestná.

Ty si dal Panne Márii plnosť milosti;

daruj nám vytrvalosť vo viere a láske.

\redrbar Nech oroduje za nás Sedembolestná.

Ty si bolestnú Pannu korunoval za Kráľovnú neba;

daj, nech sa naši zosnulí naveky radujú so všetkými svätými v tvojom kráľovstve.

\redrbar Nech oroduje za nás Sedembolestná.

%\separator{4}
%\pagebreak
%%%%%%%%%%%%%%%%%%%%%%%%%%%%%%%%%%%%%%
\vspace{2em}

\redvbar Præcéptis salutáribus móniti et divína institutióne formáti audémus dícere:

\redrbar Pater noster, qui es in cælis: 
sanctificétur nomen tuum;
advéniat regnum tuum;
fiat volúntas tua, sicut in cælo et in terra.
Panem nostrum cotidiánum da nobis hódie;
et dimítte nobis débita nostra,
sicut et nos dimíttimus debitóribus nostris;
et ne nos indúcas in tentatiónem;
sed líbera nos a malo.

\redvbar Deus, qui Fílio tuo in cruce exaltáto compatiéntem Matrem astáre voluísti, da Ecclésiæ tuæ, ut, Christi passiónis cum ipsa consors effécta, eiúsdem resurrectiónis párticeps esse mereátur. qui tecum vivit et regnat in unitáte Spíritus Sancti, Deus, per ómnia saécula saeculórum.
\redrbar Amen.

\vspace{2em}
\redvbar Dóminus vobíscum.

\redrbar Et cum spíritu tuo.

\vspace{1em}
\redvbar Sit nomen Dómini benedíctum.

\redrbar Ex hoc nunc et usque in saéculum.

\redvbar Adiutórium nostrum in nómine Dómini.

\redrbar Qui fecit cælum et terram.

\redvbar Benedícat vos omnípotens Deus, Pater, et Fílius, et Spíritus Sanctus.

\redrbar Amen.

\vspace{1em}
\redvbar Ite in pace.

\redrbar Deo grátias.
\vspace{3em}
\begin{center}\greseparator{2}{20}\end{center}

%\clearpage
\vspace*{\stretch{4}}
\begin{center}
%\begin{minipage}{.6\textwidth}
\small{Antifóny zodpovedajú pokynom \textit{ORDO CANTUS OFFICII}
\linebreak(Typis Vaticanis 2015)

Slovenské texty sa zhodujú so schváleným vydaním - slovenským prekladom \textit{Liturgia hodín podľa Rímskeho obradu} 
\linebreak(Typis Polyglottis Vaticanis, 1986)

V spolupráci s ThDr. Jozefom Kmecom, PhD.
\linebreak zostavil v roku 2023 Marek Klein (www.gregoriana.sk)}

%\end{minipage}
\end{center}
%\vspace{\stretch{1}}



%%%%%%%%%%%%%%%%%%%%%%%%%%%%%%%%%%%%%%%%%%%%%%%%%%%%%%%%%%%%%%%%%%%%%%%%%%%%%%%%%%%%%%%%%%%%%%%%

