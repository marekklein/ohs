%%%%%%%%%%%%%%%%%%%%%%%%%%%%%%%%%%%%%%%%%%%%%%%%%%%%%%%%%%%%%%%%%%%%%%%%%%%%%%%%%%%%%%%%%%%%%%%%
%%%%%%%%%%%%%%%%%%%%%%%%%%%%%%%%%%%%%%%%%%%%%%%%%%%%%%%%%%%%%%%%%%%%%%%%%%%%%%%%%%%%%%%%%%%%%%%%
\sloppy
\grechangecount{endofwordpenalty}{-2000}

\ohschapter{Vigilia Paschalis}
\pagebreak
\nadpisb{Pars prima: Sollemne initium Vigiliæ seu Lucernarium\\
Benedictio ignis et præparatio cerei}

\nadpisleft{Sacerdos et fideles signant se dum ipse dicit: In nómine Patris, et Fílii, et Spíritus Sancti, ac dein populum congregatum de more salutat eumque breviter admonet de vigilia nocturna, his vel similibus verbis:}

Fratres caríssimi, hac sacratíssima nocte, in qua Dóminus noster Iesus Christus de morte transívit ad vitam, Ecclésia invítat fílios dispérsos per orbem terrárum, ut ad vigilándum et orándum convéniant.
Si ita memóriam egérimus Páschatis Dómini, audiéntes verbum et celebrántes mystéria eius, spem habébimus participándi triúmphum eius de morte et vivéndi cum ipso in Deo.

\nadpisleft{Deinde sacerdos benedicit ignem, dicens, manibus extensis:}

Orémus.
Deus, qui per Fílium tuum claritátis tuæ ignem fidélibus contulísti, novum hunc ignem † sanctífica, et concéde nobis, ita per hæc festa paschália cæléstibus desidériis inflammári, ut ad perpétuæ claritátis puris méntibus valeámus festa pertíngere.
Per Christum Dóminum nostrum.
\Rbar. Amen.

\nadpisleft{Novo igne benedicto, unus ministrorum portat cereum paschalem ante sacerdotem, qui cum stilo incidit crucem in ipsum cereum. Deinde facit super eam litteram græcam Alpha, subtus vero litteram Omega, et inter brachia crucis quattuor numeros exprimentes annum currentem, interim dicens:}

1. Christus heri et hódie \nadpisleft{(incidit hastam erectam); }
2. Princípium et Finis 	\nadpisleft{(incidit hastam transversam); }
3. Alpha \nadpisleft{(incidit supra hastam erectam litteram Alpha); }
4. et Omega \nadpisleft{(incidit subtus hastam erectam litteram Omega). }
5. Ipsíus sunt témpora 	\nadpisleft{(incidit primum numerum anni currentis in angulo superiore sinistro crucis); }
6. et sǽcula \nadpisleft{(incidit secundum numerum anni currentis in angulo superiore dextro crucis). }
7. Ipsi glória et impérium \nadpisleft{(incidit tertium numerum anni currentis in angulo inferiore sinistro crucis); }
8. per univérsa æternitátis sǽcula. Amen \nadpisleft{(incidit quartum numerum anni currentis in angulo
inferiore dextro crucis).}

\nadpisleft{Incisione crucis et aliorum signorum peracta, sacerdos infigere potest in cereum quinque grana incensi, in modum crucis, interim dicens:}

1. Per sua sancta vúlnera 2. gloriósa 3. custódiat 4. et consérvet nos 5. Christus Dóminus. Amen.

\nadpisleft{De novo igne sacerdos accendit cereum paschalem, dicens:}

Lumen Christi glorióse resurgéntis díssipet ténebras cordis et mentis.


\nadpisb{Processio}

\nadpisleft{Cereo accenso, unus ex ministris assumit carbones ardentes de igne ac ponit eos in thuribulum et sacerdos, moro solito, incensum imponit. Diaconus vel, eo absente, alius minister idoneus, accipit a ministro cereum paschalem et ordinatur processio. Thuriferarius cum thuribulo fumiganti incedit ante diaconum vel alium ministrum, qui cereum paschalem defert. Sequuntur sacerdos cum ministris et populus, qui omnes candelas extinctas manu gestant. Ad portam ecclesiæ, diaconus, stans et elevans cereum cantat:}

\Vbar. Lumen Christi. \nadpisleft{Et omnes respondet: } \Rbar. Deo grátias.

\nadpisleft{Sacerdos accendit candelam suam de igne cerei paschalis. Deinde diaconus procedit ad medium ecclesiæ et, stans et elevans cereum, iterum cantat: }

\Vbar. Lumen Christi. \nadpisleft{Et omnes respondet: } \Rbar. Deo grátias.

\nadpisleft{Omnes candelam accendunt de igne cerei paschalis et procedunt. Diaconus, cum venerit ante altare, stans versus populum, elevat cereum et tertio cantat: }

\Vbar. Lumen Christi. \nadpisleft{Et omnes respondet: } \Rbar. Deo grátias.

\pagebreak

\nadpisleft{Deinde diaconus cereum paschalem deponit super candelabrum magnum iuxta ambonem paratum, vel in medio presbyterii. Et accenduntur lampades per ecclesiam, exceptis cereis altaris.  Cum ad altare pervenerit, sacerdos vadit ad sedem suam, candelam ministro tradit, imponit et benedicit thus sicut ad Evangelium in Missa. Diaconus adit sacerdotem et dicens:}

Iube, domne, benedícere

\nadpisleft{petit et accipit benedictionem a sacerdote dicente submissa voce:}

Dóminus sit in corde tuo et in lábiis tuis, ut digne et competénter annúnties suum paschále præcónium: in nómine Patris, et Fílii, ✠ et Spíritus Sancti. \nadpisleft{Diaconus respondet: } Amen.

\nadpisleft{Quæ benedictio omittitur, si præconium annuntiatur ab alio qui non sit diaconus. Diaconus, incensatis libro et cereo, annuntiat præconium paschale in ambone vel ad legile, omnibus stantibus et candelas accensas in manibus tenentibus. Præconium paschale annuntiari potest, absente diacono, ab ipso sacerdote vel ab alio presbytero concelebrante. Si vero, pro necessitate cantor laicus Præconium annuntiat, omittit verba Quaprópter astántes vos usque ad finem invitationis, necnon salutationem Dóminus vobíscum.}

\nadpisb{Præconii paschalis forma longior}

Exsúltet iam angélica turba cælórum:
exsúltent divína mystéria: et pro tanti Regis victória tuba ínsonet salutáris.
Gáudeat et tellus tantis irradiáta fulgóribus: et, ætérni Regis splendóre illustráta, totíus orbis se séntiat amisísse calíginem.
Lætétur et mater Ecclésia, tanti lúminis adornáta fulgóribus: et magnis populórum vócibus hæc aula resúltet.
[Quaprópter astántes vos, fratres caríssimi, ad tam miram huius sancti lúminis claritátem, una mecum, quæso, Dei omnipoténtis misericórdiam invocáte. Ut, qui me non meis méritis intra Levitárum númerum dignátus est aggregáre, lúminis sui claritátem infúndens, cérei huius laudem implére perfíciat].

[\Vbar. Dóminus vobíscum.
\Rbar. Et cum spíritu tuo.]
\Vbar. Sursum corda.
\Rbar. Habémus ad Dóminum.
\Vbar. Grátias agámus Dómino Deo nostro.
\Rbar. Dignum et iustum est.

Vere dignum et iustum est, invisíbilem Deum Patrem omnipoténtem Filiúmque eius Unigénitum, Dóminum nostrum Iesum Christum, toto cordis ac mentis afféctu et vocis ministério personáre.

Qui pro nobis ætérno Patri Adæ débitum solvit, et véteris piáculi cautiónem pio cruóre detérsit.
Hæc sunt enim festa paschália, in quibus verus ille Agnus occíditur, cuius sánguine postes fidélium consecrántur.
Hæc nox est, in qua primum patres nostros, fílios Israel edúctos de Ægýpto, Mare Rubrum sicco vestígio transíre fecísti.
Hæc ígitur nox est, quæ peccatórum ténebras colúmnæ illuminatióne purgávit.
Hæc nox est, quæ hódie per univérsum mundum in Christo credéntes, a vítiis s'éculi et calígine peccatórum segregátos, reddit grátiæ, sóciat sanctitáti.
Hæc nox est, in qua, destrúctis vínculis mortis, Christus ab ínferis victor ascéndit.
Nihil enim nobis nasci prófuit, nisi rédimi profuísset.
O mira circa nos tuæ pietátis dignátio!
O inæstimábilis diléctio caritátis: ut servum redímeres, Fílium tradidísti!
O certe necessárium Adæ peccátum, quod Christi morte delétum est!
O felix culpa, quæ talem ac tantum méruit habére Redemptórem!
O vere beáta nox, quæ sola méruit scire tempus et horam, in qua Christus ab ínferis resurréxit!
Hæc nox est, de qua scriptum est: Et nox sicut dies illuminábitur: et nox illuminátio mea in delíciis meis.
Huius ígitur sanctificátio noctis fugat scélera, culpas lavat: et reddit innocéntiam lapsis et mæstis lætítiam.
Fugat ódia, concórdiam parat et curvat impéria.

In huius ígitur noctis grátia, súscipe, sancte Pater, laudis huius sacrifícium vespertínum, quod tibi in hac cérei oblatióne sollémni, per ministrórum manus de opéribus apum, sacrosáncta reddit Ecclésia.
Sed iam colúmnæ huius præcónia nóvimus, quam in honórem Dei rútilans ignis accéndit.
Qui, licet sit divísus in partes, mutuáti tamen lúminis detriménta non novit.
Alitur enim liquántibus ceris, quas in substántiam pretiósæ huius lámpadis apis mater edúxit.
O vere beáta nox, in qua terrénis cæléstia, humánis divína iungúntur!
Orámus ergo te, Dómine, ut céreus iste in honórem tui nóminis consecrátus, ad noctis huius calíginem destruéndam, indefíciens persevéret.
Et in odórem suavitátis accéptus, supérnis lumináribus misceátur.
Flammas eius lúcifer matutínus invéniat:
Ille, inquam, lúcifer, qui nescit occásum:
Christus Fílius tuus,
qui, regréssus ab ínferis, humáno géneri serénus illúxit,
et vivit et regnat in sǽcula sæculórum.
\Rbar. Amen.
\separator{4}
%%%%%%%%%%%%%%%%%%%%%%%%%%%%%%%%%%%%%%

\nadpisb{Pars secunda: Liturgia verbi}

\nadpisleft{In hac Vigilia, matre omnium Vigiliarum, proponuntur novem lectiones, scilicet septem e Vetere Testamento et duæ e Novo (Epistola et Evangelium), quae omnes legendæ sunt ubicumque fieri potest, ut indoles Vigiliæ, quæ diuturnitatem exigit, servetur. Attamen ubi graviores circumstantiæ pastorales id postulent, minui potest numerus lectionum e Vetere Testamento; semper tamen attendatur lectionem verbi Dei esse partem fundamentalem huius Vigiliæ paschalis. Legantur saltem tres lectiones e Vetere Testamento desumptæ, et quidem ex Lege et Prophetis, et canantur respectivi Psalmi responsorii. Numquam autem omittatur lectio cap. 14 Exodi cum suo cantico. Depositis candelis, omnes sedent. Antequam incipiantur lectiones, sacerdos populum admonet, his vel similibus verbis:}

Vigíliam sollémniter ingréssi, fratres caríssimi, quiéto corde nunc verbum Dei audiámus. Meditémur, quómodo Deus pópulum suum elápsis tempóribus salvum fécerit, et novíssime nobis Fílium suum míserit Redemptórem. Orémus, ut Deus noster hoc paschále salvatiónis opus ad plenam redemptiónem perfíciat.

\nadpisleft{Deinde sequuntur lectiones. Lector ad ambonem pergit et lectionem profert. Postea psalmista seu cantor psalmum dicit, populo responsum proferente. Omnibus deinde surgentibus, sacerdos dicit Orémus, et, postquam omnes per aliquod tempus in silentio oraverint, dicit orationem lectioni respondentem. Loco psalmi responsorii servari potest spatium sacri silentii, omissa hoc in casu pausa post Orémus.}
\pagebreak
%%%%%%%%%%%%%%%%%%%%%%%%%%%%%%%%%%%%%%

%\begin{comment}
\ohslectio{le-sab-mass1}
\separator{4}

\ohspropriumcentered{Canticum}{VIII}{tr-jubilate_domino}

\ohslectio{le-sab-mass2}
\separator{4}

\ohspropriumcentered{Canticum}{VIII}{tr-qui_confidunt}

\ohslectio{le-sab-mass3}
\separator{4}

\ohspropriumcentered{Canticum}{VIII}{tr-cantemus_domino}

\ohslectio{le-sab-mass4}
\separator{4}

\ohspropriumcentered{Canticum}{VIII}{tr-laudate_dominum}

\ohslectio{le-sab-mass5}
\separator{4}

\ohspropriumcentered{Canticum}{VIII}{tr-vinea_facta_est}

\ohslectio{le-sab-mass6}
\separator{4}

\ohsproprium{Canticum}{VIII}{tr-attende_caelum}
\separator{4}

\ohslectio{le-sab-mass7}
\separator{4}

\ohspropriumcentered{Canticum}{VIII}{tr-sicut_cervus}

%%%%%%%%%%%%%%%%%%%%%%%%%%%%%%%%%%%%%%%%%%%%%%%%%%
\nadpisleft{Gloria I}
\gregorioscore{./mass/ky-gloria_i}
\separator{4}
%%%%%%%%%%%%%%%%%%%%%%%%%%%%%%%%%%%%%%%%%%%%%%%%%%

\ohslectio{le-sab-mass8}
\separator{4}

%%%%%%%%%%%%%%%%%%%%%%%%%%%%%%%%%%%%%%%%%%%%%%%%%%
\vspace*{\stretch{4}}
\greannotation{\small VIII}
\gresetinitiallines{1}
\gregorioscore{./mass/al-confitemini-1}
\vspace{1em}
\greannotation{\small \Vbar}
\gresetinitiallines{1}
\gregorioscore{./mass/al-confitemini-2}
\ohstranslate{al-confitemini}

\separator{1}
\vspace*{\stretch{4}}
\pagebreak
%%%%%%%%%%%%%%%%%%%%%%%%%%%%%%%%%%%%%%%%%%%%%%%%%%

\ohslectio{le-sab-mass9}
\separator{4}

\ohslectio{le-sab-mass10}
\separator{4}

\ohslectio{le-sab-mass11}
\separator{4}

%%%%%%%%%%%%%%%%%%%%%%%%%%%%%%%%%%%%%%
\nadpisb{Ad Liturgiam Baptismalem}
\begin{multicols}{2}
\noindent
\gresetinitiallines{0}
\vspace{-2em}
\gregorioscore{./mass/litaniae-5.gabc}
\vspace{-17pt}
\begin{tabbing}
\hspace*{0.45cm}\=\hspace*{4.04cm}\=\hspace*{1.37cm}\=\hspace*{2.05cm}\= \kill
\> Fili Redémptor mundi \> {\bf De-}us, \> mi-se-{\it ré-re}\> no-bis.\\
\> Spíritus Sancte \> {\bf De-}us, \> mi-se-{\it ré-re}\> no-bis.\\
\> Sancta Trínitas unus \> {\bf De-}us, \> mi-se-{\it ré-re}\> no-bis.
\end{tabbing}
\gresetinitiallines{0}
\gregorioscore{./mass/litaniae-6.gabc}
\vspace{-17pt}
\begin{tabbing}
\hspace*{.55cm}\=\hspace*{4.6cm}\=\hspace*{.88cm}\=\hspace*{.74cm}\= \kill
\> Sancta De- i ~ {\bf Gé}nitrix, \> ó-{\it ra} \> {\it pro} \> no-bis.\\
\> Sancta Virgo {\bf vír}ginum, \> ó-{\it ra} \> {\it pro} \> no-bis.\\
\> Sancte \qquad \quad {\bf Mí}chael, \> ó-{\it ra} \> {\it pro} \> no-bis.\\
\> Sancte \qquad \quad {\bf Gá}briel, \> ó-{\it ra} \> {\it pro} \> no-bis.\\
\> Sancte \qquad \quad {\bf Rá}phael, \> ó-{\it ra} \> {\it pro} \> no-bis.
\end{tabbing}
\gregorioscore{./mass/litaniae-7.gabc}
\vspace{-20pt}
\begin{tabbing}
\hspace*{.31cm}\=\hspace*{10.55cm}\= \kill
\> Omnes sancti beatórum Spirítuum {\bf ór}dines, \> orá{\it te~ pro}~ nobis.\' \\
\\
\> Sancte {\bf A}braham, \> o{\it ra~ pro}~ nobis.\'\\
\> Sancte He{\bf lí}a, \> o{\it ra~ pro}~ nobis.\'\\
\> Sancte {\bf Dá}niel, \> o{\it ra~ pro}~ nobis.\'\\
\> Sancte Joánnes Ba{\bf ptí}sta, \> o{\it ra~ pro}~ nobis.\' \\
\> Sancte {\bf Jo}seph, \> o{\it ra~ pro}~ nobis.\'\\
\> Omnes sancti Patriárchae et Pro{\bf phé}tae, \> orá{\it te~ pro}~ nobis.\'\end{tabbing}

\gregorioscore{./mass/litaniae-8.gabc}
\vspace{-17pt}
\begin{tabbing}
\hspace*{.23cm}\=\hspace*{10.47cm}\= \kill
\> Sancte {\bf Pau}le, \> o{\it ra pro} nobis.\'\\
\> Sancte And{\bf ré}a, \> o{\it ra pro} nobis.\'\\
\> Sancte Ja{\bf có}be, \> o{\it ra pro} nobis.\'\\
\> Sancte Jo{\bf án}nes, \> o{\it ra pro} nobis.\'\\
\> Sancte {\bf Thó}ma, \> o{\it ra pro} nobis.\'\\
\> Sancte Ja{\bf có}be, \> o{\it ra pro} nobis.\'\\
\> Sancte Phi{\bf lí}ppe, \> o{\it ra pro} nobis.\'\\
\> Sancte Bartholo{\bf máe}e, \> o{\it ra pro} nobis.\'\\
\> Sancte Ma{\bf tháe}e, \> o{\it ra pro} nobis.\'\\
\> Sancte {\bf Sí}mon, \> o{\it ra pro} nobis.\'\\
\> Sancte Thad{\bf dáe}e, \> o{\it ra pro} nobis.\'\\
\> Sancte Mat{\bf thí}a, \> o{\it ra pro} nobis.\'\\
\> Sancte {\bf Bár}naba, \> o{\it ra pro} nobis.\'\\
\> Sancte {\bf Lú}ca, \> o{\it ra pro} nobis.\'\\
\> Sancte {\bf Mar}ce, \> o{\it ra pro} nobis.\'\\
\> Omnes sancti Apóstoli et Evange{\bf lí}stae, \> orá{\it te pro} nobis.\'\\
\> Omnes sancti discípuli {\bf Dó}mini, \> orá{\it te pro} nobis.\'\\
\\
\> Omnes sancti Inno{\bf cén}tes, \> orá{\it te pro} nobis.\'\\
\> Sancte {\bf Sté}phane, \> o{\it ra pro} nobis.\'\\
\> Sancte Lau{\bf rén}ti, \> o{\it ra pro} nobis.\'\\
\> Sancte Vin{\bf cén}ti, \> o{\it ra pro} nobis.\'\\
\> Sancti Fabiáne et Sebasti{\bf á}ne, \> orá{\it te pro} nobis.\'\\
\> Sancti Joánnes et {\bf Páu}le, \> orá{\it te pro} nobis.\'\\
\> Sancti Cosma et Dami{\bf á}ne, \> orá{\it te pro} nobis.\'\\
\> Sancti Gervási et Pro{\bf tá}si, \> orá{\it te pro} nobis.\'\\
\> Sancte Cypri{\bf á}ne, \> o{\it ra pro} nobis.\'\\
\> Sancte Sta{\bf ní}slae, \> o{\it ra pro} nobis.\'\\
\> Sancte Boni{\bf fá}ti, \> o{\it ra pro} nobis.\'\\
\> Omnes sancti {\bf Már}tyres, \> orá{\it te pro} nobis.\'\\
\\
\> Sancte Mar{\bf tí}ne, \> o{\it ra pro} nobis.\'\\
\> Sancte Augus{\bf tí}ne, \> o{\it ra pro} nobis.\'\\
\> Sancte Gre{\bf gó}ri, \> o{\it ra pro} nobis.\'\\
\> Sancte Am{\bf bró}si, \> o{\it ra pro} nobis.\'\\
\> Sancte Hie{\bf ró}nyme, \> o{\it ra pro} nobis.\'\\
\> Sancte Sil{\bf vé}ster, \> o{\it ra pro} nobis.\'\\
\> Sancte Nico{\bf lá}ë, \> o{\it ra pro} nobis.\'\\
\> Sancte Atha{\bf ná}si, \> o{\it ra pro} nobis.\'\\
\> Sancte Basíli et Gre{\bf gó}ri, \> o{\it ra pro} nobis.\'\\
\> Sancte Joánnes Chry{\bf só}stome, \> o{\it ra pro} nobis.\'\\
\> Sancti Cyrílle et Me{\bf thó}di, \> orá{\it te pro} nobis.\'\\
\> Sancte Pa{\bf trí}ci, \> o{\it ra pro} nobis.\'\\
\> Sancte {\bf Cá}role, \> o{\it ra pro} nobis.\'\\
\> Sancte {\bf Pí}e, \> o{\it ra pro} nobis.\'\\
\> Omnes sancti Pontífices et Confes{\bf só}res, \> orá{\it te pro} nobis.\'\\
\> Omnes sancti Do{\bf ctó}res, \> orá{\it te pro} nobis.\'\\
\\
\> Sancte An{\bf tó}ni, \> o{\it ra pro} nobis.\'\\
\> Sancte Bene{\bf dí}cte, \> o{\it ra pro} nobis.\'\\
\> Sancte Ber{\bf nár}de, \> o{\it ra pro} nobis.\'\\
\> Sancte Nor{\bf bér}te, \> o{\it ra pro} nobis.\'\\
\> Sancte Do{\bf mí}nice, \> o{\it ra pro} nobis.\'\\
\> Sancte Fran{\bf cí}sce, \> o{\it ra pro} nobis.\'\\
\> Sancte {\bf Bru}no, \> o{\it ra pro} nobis.\'\\
\> Sancte {\bf Sté}phane, \> o{\it ra pro} nobis.\'\\
\> Sancte Gugli{\bf él}me, \> o{\it ra pro} nobis.\'\\
\> Omnes sancti Sacerdótes et Le{\bf ví}tae, \> orá{\it te pro} nobis.\'\\
\> Omnes sancti Mónachi et Ere{\bf mí}tae, \> orá{\it te pro} nobis.\'\\
\\
\> Sancte Ludo{\bf ví}ce, \> o{\it ra pro} nobis.\'\\
\> Sancte Hen{\bf rí}ce, \> o{\it ra pro} nobis.\'\\
\> Omnes sancti La{\bf í}ci, \> orá{\it te pro} nobis.\'\\
\\
\> Sancta María Magda{\bf lé}na, \> o{\it ra pro} nobis.\'\\
\> Sancta {\bf A}gatha, \> o{\it ra pro} nobis.\'\\
\> Sancta {\bf Lú}cia, \> o{\it ra pro} nobis.\'\\
\> Sancta {\bf A}gnes, \> o{\it ra pro} nobis.\'\\
\> Sancta Cae{\bf cí}lia, \> o{\it ra pro} nobis.\'\\
\> Sancta {\bf Mó}nica, \> o{\it ra pro} nobis.\'\\
\> Sancta E{\bf lí}sabeth, \> o{\it ra pro} nobis.\'\\
\> Sancta Catha{\bf rí}na, \> o{\it ra pro} nobis.\'\\
\> Sancta Ana{\bf stá}sia, \> o{\it ra pro} nobis.\'\\
\> Sancta Scho{\bf lá}stica, \> o{\it ra pro} nobis.\'\\
\> Sancta Petro{\bf ní}lla, \> o{\it ra pro} nobis.\'\\
\> Sancta Hilde{\bf gár}dis, \> o{\it ra pro} nobis.\'\\
\> Sancta Ger{\bf trú}dis, \> o{\it ra pro} nobis.\'\\
\> Sancta Te{\bf ré}sia, \> o{\it ra pro} nobis.\'\\
\> Omnes sanctae Vírgines et {\bf Ví}duae, \> orá{\it te pro} nobis.\'\\
\\
\> Sancte {\bf Clé}mens, \> o{\it ra pro} nobis.\'\\
\> Sancte {\bf Clé}te, \> o{\it ra pro} nobis.\'\\
\> Sancte {\bf Lí}ne, \> o{\it ra pro} nobis.\'\\
\> Sancte Mar{\bf cé}lle, \> o{\it ra pro} nobis.\'\\
\> Sancte Cal{\bf lí}ste, \> o{\it ra pro} nobis.\'\\
\> Sancte Ponti{\bf á}ne, \> o{\it ra pro} nobis.\'\\
\> Sancte {\bf Clé}mens, \> o{\it ra pro} nobis.\'\\
\> Sancte Apolli{\bf ná}ris, \> o{\it ra pro} nobis.\'\\
\> Sancte {\bf E}phrem, \> o{\it ra pro} nobis.\'\\
\> Sancte Petre Chry{\bf só}loge, \> o{\it ra pro} nobis.\'\\
\> Sancte Hi{\bf lá}ri, \> o{\it ra pro} nobis.\'\\
\> Sancte {\bf Bé}da, \> o{\it ra pro} nobis.\'\\
\> Sancte {\bf Lé}o, \> o{\it ra pro} nobis.\'\\
\> Sancte Boni{\bf fá}ti, \> o{\it ra pro} nobis.\'\\
\> Sancte Bonaven{\bf tú}ra, \> o{\it ra pro} nobis.\'\\
\> Sancte Caje{\bf tá}ne, \> o{\it ra pro} nobis.\'\\
\> Sancte Jus{\bf tí}ne, \> o{\it ra pro} nobis.\'\\
\> Sancte An{\bf sel}me, \> o{\it ra pro} nobis.\'\\
\> Sancte {\bf Hér}vae, \> o{\it ra pro} nobis.\'\\
\> Sancte Ludovíce Ma{\bf rí}a, \> o{\it ra pro} nobis.\'\\
\> Sancte {\bf O}do, \> o{\it ra pro} nobis.\'\\
\> Sancte {\bf Má}jole, \> o{\it ra pro} nobis.\'\\
\> Sancte O{\bf dí}lo, \> o{\it ra pro} nobis.\'\\
\> Sancte {\bf Hu}go, \> o{\it ra pro} nobis.\'\\
\> Sancte Romu{\bf ál}de, \> o{\it ra pro} nobis.\'\\
\> Sancte Ig{\bf ná}ti, \> o{\it ra pro} nobis.\'\\
\> Sancte {\bf Mau}re, \> o{\it ra pro} nobis.\'\\
\> Sancte Pau{\bf lí}ne, \> o{\it ra pro} nobis.\'\\
\> Sancte Ger{\bf má}ne, \> o{\it ra pro} nobis.\'\\
\> Sancte {\bf Ro}che, \> o{\it ra pro} nobis.\'\\
\> Sancti Hippólyte et Casi{\bf á}ne, \> orá{\it te pro} nobis.\'\\
\> Sancti Plácide cum sóciis {\bf tu}is, \> orá{\it te pro} nobis.\'\\
\> Sancti Juste et {\bf Pas}tor, \> orá{\it te pro} nobis.\'\\
\> Sancte Juli{\bf á}ne, \> o{\it ra pro} nobis.\'\\
\> Sancte Johánne Ma{\bf rí}a, \> o{\it ra pro} nobis.\'\\
\> Sancte Maximili{\bf á}ne, \> o{\it ra pro} nobis.\'\\
%\> Sancti Candide et Valentine, \\
%\> \hspace{1em} benedicte protectores {\bf no}stri. \> orá{\it te pro} nobis.\'\\
\> Omnes Sancti et Sanctae {\bf De}i, \> orá{\it te pro} nobis.\'
\end{tabbing}

%\pagebreak
\gregorioscore{./mass/litaniae-9.gabc}
\vspace{-11pt}
\begin{tabbing}
\hspace*{.01cm}\=\hspace*{8.2cm}\= \kill
\> Propi\textit{tius} \textbf{e}sto,  \> Exaudi nos Domine.\\
\> Ab \textit{omni} \textbf{ma}lo, \> líbera nos Dómine.\\
\> Ab om\textit{ni pec}\textbf{cá}to, \> líbera nos Dómine.\\
\> Ab insidi\textit{is di}\textbf{á}boli, \> líbera nos Dómine.\\
\> Ab ira, et ódio, et omni mala \textit{volun}\textbf{tá}te, \> líbera nos Dómine.\\
\> A spiritu forni\textit{cati}\textbf{ó}nis, \> líbera nos Dómine.\\
\> A fúlgure et \textit{tempe}\textbf{stá}te, \> líbera nos Dómine.\\
\> A flagéllo \textit{terrae}\textbf{mó}tus, \> líbera nos Dómine.\\
\> A peste, fa\textit{me et} \textbf{bel}lo, \> líbera nos Dómine.\\
\> A mor\textit{te per}\textbf{pé}tua, \> líbera nos Dómine.\\
\> Per mystérium sanctae incarnati\textit{ónis} \textbf{tu}ae, \> líbera nos Dómine.\\
\> Per advén\textit{tuum} \textbf{tu}um, \> líbera nos Dómine.\\
\> Per nativi\textit{tátem} \textbf{tu}am, \> líbera nos Dómine.\\
\> Per baptismum et sanctum jejú\textit{nium} \textbf{tu}um, \> líbera nos Dómine.\\
\> Per crucem et passi\textit{ónem} \textbf{tu}am, \> líbera nos Dómine.\\
\> Per mortem et sepul\textit{túram} \textbf{tu}am, \> líbera nos Dómine.\\
\> Per sanctam resurrecti\textit{ónem} \textbf{tu}am, \> líbera nos Dómine.\\
\> Per admirábilem ascensi\textit{ónem} \textbf{tu}am, \> líbera nos Dómine.\\
\> per advéntum Spiritus San\textit{cti Pa}\textbf{rá}cliti, \> líbera nos Dómine.\\
\> In di\textit{e ju}\textbf{di}cii, \> líbera nos Dómine.
\end{tabbing}
\end{multicols}

\gresetinitiallines{0}
\gregorioscore{./mass/litaniae-3.gabc}
Ut no\textit{bis} \textbf{par}cas, \hfill te rogámus, audi nos.\\
Ut nobis \textit{in}\textbf{dúl}geas, \hfill te rogámus, audi nos.\\
Ut ad veram paeniténtiam nos perdúcere \textit{di}\textbf{gné}ris, \hfill te rogámus, audi nos.\\
Ut Ecclésiam tuam sanctam régere et conserváre \textit{di}\textbf{gné}ris, \hfill te rogámus, audi nos.\\
Ut domnum Apostólicum et omnes ecclesiasticos ór\underline{dines}~† in sancta religióne conserváre \textit{di}\textbf{gné}ris, \hfill te rogámus, audi nos.\\
Ut inimicos sanctae Ecclésiae humiliáre \textit{di}\textbf{gné}ris, \hfill te rogámus, audi nos.\\
Ut régibus et principibus christiánis pacem, et veram concórdiam donáre \textit{di}\textbf{gné}ris, \hfill te rogámus, audi nos.\\
Ut cuncto pópulo christiáno pacem et unitátem largiri \textit{di}\textbf{gné}ris, \hfill te rogámus, audi nos.\\
Ut omnes errántes ad unitátem Ecclésiae revocá\underline{re},~† et infidéles univérsos ad Evangélii lumen perdúcere \textit{di}\textbf{gné}ris. \hfill te rogámus, audi nos.\\
Ut nosmetipsos in tuo sancto servítio confortáre, et conserváre \textit{di}\textbf{gné}ris, \hfill te rogámus, audi nos.\\
Ut mentes nostras ad caeléstia desidéri\textit{a} \textbf{é}rigas, \hfill te rogámus, audi nos.\\
Ut ómnibus benefactóribus nostris sempitérna bona \textit{re}\textbf{trí}buas, \hfill te rogámus, audi nos.\\
Ut ánimas nostras, fratrum, propinquórum et benefactórum ab aetérna damnatióne \textit{e}\textbf{ri}pias, \hfill te rogámus, audi nos.\\
Ut fructus terrae dare et conserváre \textit{di}\textbf{gné}ris, \hfill te rogámus, audi nos.\\
Ut ómnibus fidélibus defúnctis réquiem aetérnam donáre \textit{di}\textbf{gné}ris, \hfill te rogámus, audi nos.\\
Ut nos exaudire \textit{di}\textbf{gné}ris \hfill te rogámus, audi nos.\\
Fi\textit{li} \textbf{De}i, \hfill te rogámus, audi nos.
%Agnus Dei, qui tollis peccáta mundi, parce nobis, Dómine.
%Agnus Dei, qui tollis peccáta mundi, Dómine,
%Agnus Dei, qui tollis peccáta mundi, exáudi nos, Dómine,

\gregorioscore{./mass/litaniae-4.gabc}



\separator{4}
%%%%%%%%%%%%%%%%%%%%%%%%%%%%%%%%%%%%%%
\gresetinitiallines{1}
\ohspropriumcentered{Ant.}{VIII}{an-vidi_aquam}
%\separator{4}
%%%%%%%%%%%%%%%%%%%%%%%%%%%%%%%%%%%%%%

\ohspropriumcentered{Offertorium}{II}{of-dextera_domini}

\gresetinitiallines{1}
%%%%%%%%%%%%%%%%%%%%%%%%%%%%%%%%%%%%%%
%\begin{comment}



\ifx\ohsprintall\undefined
%	\separator{4}
\else
%	\separator{4}
	\nadpisleft{Sanctus I}
	\gregorioscore{./mass/ky-sanctus_i}
	\vspace{1em}

	\separator{4}
	%\pagebreak
	\nadpisleft{Agnus I}
	\gregorioscore{./mass/ky-agnus_i}
\fi
%\end{comment}

\separator{4}
%%%%%%%%%%%%%%%%%%%%%%%%%%%%%%%%%%%%%%
\gregorioscore{./mass/alleluia}
\gresetinitiallines{0}
%\gresetnabcfont{gregall}{12}
\grechangestaffsize{12}%
\grechangedim{spaceabovelines}{1mm}{scalable}%
\gregorioscore{./mass/co-verses-pascha_nostrum}
%%%%%%%%%%%%%%%%%%%%%%%%%%%%%%%%%%%%%%

%\vspace{1em}
\begin{center}\greseparator{2}{20}\end{center}
%%%%%%%%%%%%%%%%%%%%%%%%%%%%%%%%%%%%%%%%%%%%%%%%%%%%%%%%%%%%%%%%%%%%%%%%%%%%%%%%%%%%%%%%%%%%%%%%
%%%%%%%%%%%%%%%%%%%%%%%%%%%%%%%%%%%%%%%%%%%%%%%%%%%%%%%%%%%%%%%%%%%%%%%%%%%%%%%%%%%%%%%%%%%%%%%%

